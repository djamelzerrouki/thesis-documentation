\documentclass{article}
\usepackage[utf8]{inputenc}
\usepackage[english]{babel}

\usepackage{comment}

\usepackage[natbib = false,
backend = bibtex8,
style = authoryear, 
maxcitenames = 2,
mincitenames = 1,
maxbibnames = 100
]{biblatex}

\addbibresource{main-blx.bib}

\title{Bibliography management: \texttt{biblatex} package}
\author{Overleaf}
\date{ }

\begin{document}
	
	\maketitle
	
	Using \texttt{biblatex} you can display bibliography divided into sections, 
	depending of citation type. 
	Let's cite! The Einstein's journal paper \cite{einstein} and the Dirac's 
	book \cite{dirac} are physics related items. 
	Next, \textit{The \LaTeX\ Companion} book \cite{latexcompanion}, the Donald 
	Knuth's website \cite{knuthwebsite}, \textit{The Comprehensive Tex Archive 
		Network} (CTAN) \cite{ctan} are \LaTeX\ related items; but the others Donald 
	Knuth's items \cite{knuth-fa,knuth-acp} are dedicated to programming. 
	
	\medskip
	
	\printbibliography[title={Whole bibliography}]
	
\end{document}