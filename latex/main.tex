\documentclass[12pt,twoside]{report}
\usepackage[french]{babel}
\usepackage[utf8]{inputenc}
\usepackage{pdfpages}
\usepackage{amsmath,amsfonts,amssymb,amsthm,epsfig,epstopdf,titling,url,array}
\usepackage{acronym}   	 	% list of abbreviations / acronyms
\usepackage{makeidx}
\makeindex

\usepackage{arabtex}
\usepackage[LFE,LAE]{fontenc}
 

 
\usepackage{float}
\usepackage{listings}
\usepackage[acronym]{glossaries}
\usepackage[toc,page]{appendix}


\usepackage[T1]{fontenc}
\usepackage{inconsolata}

\usepackage{color}
\definecolor{gray}{rgb}{0.4,0.4,0.4}
\definecolor{darkblue}{rgb}{0.0,0.0,0.6}
\definecolor{cyan}{rgb}{0.0,0.6,0.6}
\definecolor{pblue}{rgb}{0.13,0.13,1}
\definecolor{pgreen}{rgb}{0,0.5,0}
\definecolor{pred}{rgb}{0.9,0,0}
\definecolor{pgrey}{rgb}{0.46,0.45,0.48}

 
\lstdefinelanguage{XML}
{
	 	 frame=lines, 
		breaklines=true,
	morestring=[b]",
	morestring=[s]{>}{<},
	morecomment=[s]{<?}{?>},
	stringstyle=\color{black},
	identifierstyle=\color{darkblue},
	keywordstyle=\color{cyan},
	morekeywords={xmlns,version,type}% list your attributes here
}
\theoremstyle{plain}
\newtheorem{thm}{Theorem}[section]
\newtheorem{lem}[thm]{Lemma}
\newtheorem{prop}[thm]{Proposition}
\newtheorem*{cor}{Corollary}
\usepackage{wrapfig}
\usepackage{graphicx}

\newcommand*{\authorimg}[1]{%
	\raisebox{-.3\baselineskip}{%
		\includegraphics[width=0.1\textwidth]{#1}%
	}%
}
\graphicspath{{images/}}
\usepackage[a4paper,width=150mm,top=25mm,bottom=25mm,bindingoffset=6mm]{geometry}
\usepackage{caption}
\usepackage{subcaption}
\usepackage{fancyhdr}
\pagestyle{fancy}

%\fancyhead{}
%\fancyhead[LO,CE]{Chapter \thechapter}
%\fancyhead[RO,LE]{Conception et mise en place d’un système Workflow  pour l’envirenement cloud}
%\fancyfoot{}
%\fancyfoot[LE,RO]{\thepage}
%\fancyfoot[LO,CE]{Chapter \thechapter}
%\fancyfoot[CO,RE]{Zerrouki djamel}
\renewcommand{\headrulewidth}{0.4pt}
\renewcommand{\footrulewidth}{0.4pt}

\usepackage{array}
%\usepackage[table,xcdraw]{xcolor}
%\setcounter{tocdepth}{4}
%\setcounter{secnumdepth}{4}





\newcommand{\comment}[1]{}

%opening
\usepackage[natbib = false,
            backend = bibtex8,
            style = authoryear, 
            maxcitenames = 2,
            mincitenames = 1,
            maxbibnames = 100
            ]{biblatex}
            
            
 
 
 \theoremstyle{definition}
 \newtheorem{defn}{Définition}[section]
 \newtheorem{conj}{Conjecture}[section]
 \newtheorem{exmp}{Example}[section]
  \newtheorem{att}{Attention}[section]
 \theoremstyle{Remarque}
 \newtheorem{rem}{Remarque}[section]
 \newtheorem*{note}{Note}
%\renewcommand{\thesection}{\Alph{section}}
%\renewcommand{\thesubsection}{\Roman{section} - \arabic{subsection}}
\addbibresource{references.bib}
%\title{
%	{\textbf{Conception et mise en place d’un %système Workflow  pour l'environnement cloud}} \\   
%	{$ Instituation $}\\
%	{\includegraphics{logof.PNG}}
%	{\includegraphics{univtiaret.png}}
%}
\author{Zerrouki Djamel}
 
\begin{document}


%\maketitle
% \begin{titlepage}
    \begin{center}
        \vspace*{1cm}
 
        \Huge
        \textbf{Conception et mise en place d’un système Workflow  pour l'environnement cloud}
 
        \vspace{0.5cm}
        \LARGE
        Workflow et Cloud
 
        \vspace{1.5cm}
 Thèse présentée par\\
        \textbf{Zerrouki Djamel}
 
        \vspace{0.5cm}
 
      Encadre par:  \textbf{M.Lahcen aid}
       
        \vfill
 
        Une thèse présentée pour le diplôme de \\
         Master 2  en Génie  logiciel
 
        \vspace{0.8cm}
 
        \includegraphics[width=0.4\textwidth]{univtiaret}
 
        \Large
        Département Informatique\\
        Université IBN KHALDOUN\\
        Tiaret\\
        2018-2019
 
    \end{center}
\end{titlepage}
              \includepdf{pagedegarde.pdf}

               \includepdf{pagedegarde.pdf}
 
 
 




\clearpage

\printglossary[title=Special Terms, toctitle=List of terms]
 
\pagenumbering{roman}
 
 \includepdf{jimmi.pdf}
   \chapter*{Dédicace}
\begin{center}
	\textit{  Je dédie ce modeste travail :}
\end{center}
  \begin{center}
  	\textit{\textbf{ À ma mère:} }
  \end{center}
 \begin{center}
 	\textit{ Nulle dédicace ne saurait exprimer suffisamment ma gratitude, mon amour et mon profond respect à ma mère. Pour son soutien, son amour et son sacrifice. Sa présence et ses encouragements sont pour moi les piliers fondateurs de ce que je suis et de ce que je fais.}
 \end{center}
  \begin{center}
  	\textit{\textbf{À mon père :} }
  \end{center}
 \begin{center}
 	\textit{  Puisque rien au monde ne pourrait compenser les sacrifices démesurés qu’il a déployés pour guider mes pas, et ses encouragements continus. Que mon père accepte à cette occasion, mes hommages comme gage de mon profond amour, et ma reconnaissance jamais interrompue.}
 \end{center}
 
\begin{center}
	\textit{ \textbf{À mes chers sœurs Habiba et Farah,}}
\end{center}

\begin{center}
	\textit{\textbf{À mes très chères tantes, oncles, cousins et cousines,}}
\end{center}

\begin{center}
	\textit{ \textbf{À toute ma grande Famille,} }
\end{center}


\begin{center}
	\textit{ \textbf{À Mes Amis,} }
\end{center}

\begin{center}
	\textit{particulièrement :    Ilyasse, Djamel, Sofian, Walid, Abdelhak, Yacine, Mohamed,  Houari, Khaled, Chahra . }
\end{center}

\begin{center}
	\textit{\textbf{Et à tous ceux qui ont été présent pour moi tout au long de mes études, et à toutes les personnes que j’apprécie et que je n’ai pas citées.}}
\end{center}
   
  
 
 \chapter*{Remerciements}
 \large

\textit{	Avant de présenter notre modeste travail, il nous apparaît opportun de présenter d’abord nos remerciements : }
 
\textit{Avant tout, nous remercions \textbf{ALLAH}, le tout puissant qui nous a donnés à la fois, le courage et la patience pour bien mener ce travail.}
 
 
\textit{	Nous manifestons nos remerciements distingués à notre encadreur \textbf{M.LAHCEN Aid} pour tous ses efforts fournis, son aide immense, ses conseils précieux et ses orientations pour atteindre nos objectifs. }


\textit{Nous adressons nos plus profondes reconnaissances à la communauté GitHub pour les bibliothèques open sources que nous avons utilisées, à StackOverFlow pour leurs réponses à nos questions techniques et à CodeAcademy,sans oublier tous les  youtubers qui nous ont offert des formations  gratuitement. 
}

\textit{Et nous remercions les membres de jurys qui ont accepté d’évaluer ce travail de fin d’études}. 


\textit{Sans oublier, toute personne ayant contribué, de près ou de loin, à l’accomplissement de ce travail.  }


\textit{En espérant que ce modeste travail soit à la hauteur et reflète ce que nous avons pu acquérir pendant la durée du projet. }
 
\normalsize


 
 
 
\chapter*{Résumé}

\large

La technologie de Workflow découle du besoin des entreprises d’améliorer leurs performances, notamment le contrôle des informations, une meilleure collaboration entre les parties prenantes et de meilleurs processus commerciaux et opérationnels.

L'objectif de notre projet est de concevoir et de réaliser un système Workflow administrative  pour environnement Cloud.
 

Afin de répondre à cet objectif,  nous avons développé un flux de travail administratif pour l'environnement Cloud, permettant l'exécution et le fonctionnement automatiques des processus organisationnels afin de formaliser, normaliser et accélérer le traitement des documents et autres tâches.

Pour réaliser ce travail, nous avons dû mener une étude théorique pour comprendre le domaine de Cloud  et Workflow  puis, utiliser les informations de cette étude pour compléter l'analyse, conception et l'implémentation de système.



\textbf{\underline{Mots clés:}}    Informatique en nuage (Cloud Computing), Flux de travail (Workflow),Processus Métier, Réseaux de Flux de travail, Spring Framework, Micro-Services. 


\normalsize

 

\chapter*{Abstrait}

Workflow technology is driven by the need for businesses to improve their performance, including information control, better collaboration between stakeholders, and better business and operational processes.

The goal of our project is to design and build an administrative workflow system for the cloud environment.


In order to meet this objective, we have developed an administrative workflow for the cloud environment, enabling the automatic execution and operation of organizational processes to formalize, standardize and speed up the processing of documents and other tasks.

To perform this work, we had to conduct a theoretical study to understand the cloud domain and workflow then use the information from this study to complete the analysis, design and system implementation.


\textbf{Keywords:} Cloud, workflow, business process.
  

\includepdf{molakass.pdf}


  
 

%------------ List of Abbreviations---------------------


	\pagenumbering{roman} % switch to roman numerals

%------------ Introduction Général ---------------------

 



%%%%%%%  list of Contents %%%%%%%
\tableofcontents

%%%%%%%  list of Figures %%%%%%%
\listoffigures

%%%%%%%  list of Tables %%%%%%%
\listoftables
%%%%%%%  Abbreviations %%%%%%%
 
 \chapter*{Liste Des Abréviations}
 \begin{acronym}
 		\acro{als}[ALS]{Service Level Agreement} .	 
 		\acro{aws}[AWS]{Amazon Web Services}
 		
 		    \acro{bpaas}[BPaaS]{Business Process as-a-Service}	 
 		\acro{bpr}[BPR]{Business Process Reengineering} 
 		
 		   \acro{cnr}[CNR]{Caisse Nationale Des Retraites}
 			\acro{cpu}[CPU]{Central Processing Unit}
 		
 \acro{ec2}[EC2]{Elastic Compute cloud}.
 		
 		  	\acro{gae}[GAE]{Google App Engine}	
 		
 		\acro{haas}[HaaS]{Hardware as-a-Service}
 		\acro{iaas}[IaaS]{Infrastructure as-a-Service}	
 	\acro{ibm}[IBM]{International Business Machines}	 .
 	
 	
 	
 	    \acro{nist}[NIST]{National Institute of Standards and Technology} 	
 	    			\acro{paas}[PaaS]{Platform as-a-Service}	
 	   \acro{qos}[QoS]{Quality Of Service}	 
 	   	\acro{saas}[SaaS]{Software as-a-Service}	
 	   	  \acro{sgbd}[SGBD]{Système de Gestion de Bases de Données}

    	
    	
 \acro{uims}[UIMS]{ User Interface Management Systems}
 	       \acro{uml}[UML]{Unified Modeling Language}

 	    	       \acro{up}[UP]{Unified Process}
\acro{vmm}[VMM]{Virtual Machine Monitor}  

 	  \acro{vpn}[VPN]{Virtual Private Network}
\acro{wapi}[WAPI]{Workflow Application Programming interface}
    	% WORKFLOW
    	\acro{wfms}[WFMS]{WorkFlow Management System}
    	
    	
    	\acro{wfmc}[WfMC]{Workflow Management Coalition }
    	\acro{wfmc-tc}[WFMC-TC]{Workflow Management Coalition Terminology and Glossary}		
    
	\acro{xaas}[XaaS]{X as-a-Service}	

 	  
 	  
 	  
 	  
 
 	       	  


 	       \acro{}[]{}
 	       \acro{}[]{}
 	       \acro{}[]{}
 	       \acro{}[]{}
 	       \acro{}[]{}
 	       \acro{}[]{}
 	       \acro{}[]{}
 	       \acro{}[]{}
 	       \acro{}[]{}
 	       \acro{}[]{}
 	       \acro{}[]{}
 	      
 	       
 \end{acronym}
 
 \pagenumbering{arabic}
\chapter*{Introduction Général }


\section*{Contexte et Motivation} 
Le cloud computing  présente une technologie prometteuse qui facilite l'exécution des applications . Il fournit des services flexibles et évolutifs, à la demande des utilisateurs, via un modèle de paiement à l'usage. Généralement, il peut fournir quatre types de services: SaaS (Software as a Service), PaaS (Platform as a Service), IaaS (Infrastructure as a Service) et HaaS (Hardware as a Service). Ces services sont offerts avec différents niveaux de qualité de service afin de répondre aux besoins spécifiques de différents utilisateurs. Bien que de nombreux services de cloud computing  ont des fonctionnalités similaires (par exemple des services de calculs, services de stockages, services  réseaux, etc.),

Parmi ces applications, qui peuvent être  utilisées en tant que service cloud, se trouvent des applications de flux de travaux  qui permettent l'exécution et le fonctionnement automatisés des processus des organisations pour formaliser, normaliser et accélérer le traitement des documents et autres tâches.

 Les documents et les dossiers en attente sur le bureau des collaborateurs ou dans la boîte aux lettres électronique pour examen et approbation sont des raisons courantes de retards et de goulots d'étranglement. Les opérations sont supervisées et les décisions peuvent être suivies à des fins d'audit. Le flux de travail des affaires électroniques fournit aux organisations les outils nécessaires pour exécuter les opérations plus rapidement, obtenir une meilleure visibilité et une plus grande transparence, tout en garantissant la conformité avec les éléments du journal d'audit automatisé.


\section*{Problématique}


 
\section*{Objectifs}


\section*{Organisation du mémoire}


 
 
 
 
% \part{ Étude bibliographique }
%\appendix
\fancyhead{}
\chapter{Concepts fondamentaux de cloud et de workflow }
 	 \section{Introduction}
 	  
    
    Le cloud computing, traduit le plus souvent en français par " informatique dans les nuages", " informatique dématérialisée " ou encore " infonuagique ", est un domaine qui regroupe un ensemble de techniques et de pratiques consistant à accéder, en libre-service, à du matériel ou à des logiciels informatiques, à travers une infrastructure réseau (Internet). Ce concept rend possible la distribution des ressources informatiques sous forme de services pour lesquels l'utilisateur paie uniquement pour ce qu'il utilise. Ces services peuvent être utilisés pour exécuter des applications scientifiques et commerciales, souvent modélisées sous forme de workflows.
     
    Ce chapitre présente une introduction au cloud computing et au workflow, nécessaire pour la compréhension générale de ce rapport.
    
    Tout d’abord, nous présentons dans la section 1.2 une introduction au paradigme du cloud computing. Nous donnons un aperçu général du cloud computing, y compris sa définition, ses caractéristiques principales et une comparaison avec les technologies connexes. Nous présentons les différents modèles de service, les différents modèles de déploiement, ainsi que les différents acteurs du cloud computing. Nous résumons quelques challenges de recherche en cloud computing. Par la suite, nous présentons, dans la section 1.3, une introduction au workflow et systèmes de gestion de workflow. Nous donnons le concept du workflow, sa définition, et l’architecture de référence d’un système de gestion de workflows et, finalement, nous résumons l'intérêt du cloud pour les workflows.
    
    \section{cloud computing}
    \subsection{Concept du cloud computing}
  L’idée principale du cloud est apparue dans les années 60, où le professeur John McCarthy avait imaginé que les ressources informatiques seront fournies comme des services d’utilité publique (Garfinkel, 1999). C'est ensuite, vers la fin des années 90, que ce concept a pris de l'importance avec l’avènement du grid computing  (Foster, 1999). Le terme cloud est une métaphore exprimant la similarité avec le réseau électrique, dans lequel l'électricité est produite dans de grandes centrales, puis disséminée à travers un réseau jusqu'aux utilisateurs finaux. Ici, les grandes centrales sont les Datacenter, le réseau est le plus souvent celui d'Internet et l'électricité correspond aux ressources informatiques. Le cloud computing  n'est véritablement apparu qu'au cours de l’année 2006 (Vouk, 2008) avec l'apparition d'Amazon EC2 (Elastic Compute cloud). C'est en 2009 que la réelle explosion du cloud survint avec l'arrivée sur le marché de sociétés comme Google (Google App Engine), Microsoft (Microsoft Azure), IBM (IBM Smart Business Service), Sun (Sun cloud) et Canonical Ltd (Ubuntu Enterprise cloud). D'après une étude menée par Forrester (Ried, 2011), le marché du cloud computing  s'élevait à environ 5,5 milliards de dollars en 2008, il devrait atteindre plus de 150 milliards d'ici 2020, comme l’illustre la figure \ref{fig:tempsnip4}. 
    
    \begin{figure}[h]
    	\centering
    	\includegraphics[width=0.7\linewidth]{images/tempsnip4}
    	\caption{Prévisions de la taille du marché du cloud computing  public (Ried, 2011).}
    	\label{fig:tempsnip4}
    \end{figure}
\subsubsection{Vers une définition du cloud computing }
Beaucoup de chercheurs ont tenté de définir le cloud computing (Geelan, 2008 ; McFedries, 2008 ; Buyya, 2009 ; Armbrust, 2010). La plupart des définitions attribuées à ce concept semblent se concentrer seulement sur certains aspects technologiques. L'absence d'une définition standard a généré non seulement des exagérations du marché, mais aussi des confusions. Pour cette raison, il y a eu récemment des travaux sur la normalisation de la définition du cloud computing, à l'exemple de Vaquero et coll (Vaquero, 2009) qui ont comparé plus de 20 définitions différentes et ont proposé une définition globale.  En guise de synthèse des différentes propositions données dans la littérature, nous introduisons une définition mixte, qui correspond aux différents types de cloud considérés dans les travaux réalisés dans cette thèse.

  Nous définissons le cloud comme un modèle informatique qui permet d’accéder, d’une façon transparente et à la demande, à un pool de ressources hétérogènes physiques ou virtualisées (serveurs, stockage, applications et services) à travers le réseau. Ces ressources sont délivrées sous forme de services reconfigurables et élastiques, à base d’un modèle de paiement à l’usage, dont les garanties sont offertes par le fournisseur via des contrats de niveau de service (SLA, Service Level Agreement).     

    \subsubsection{Caractéristiques principales du cloud computing}
    Le cloud computing  possède les caractéristiques suivantes :
    \begin{itemize}
    	\item 	\textbf{Accès en libre-service à la demande}. Le cloud computing offre des ressources et services aux utilisateurs à la demande. Les services sont fournis de façon automatique, sans nécessiter d’interaction humaine (Mell, 2011). 
    \item	\textbf{Accès réseau universel.}  Les services de cloud computing  sont facilement accessibles au travers du réseau, par le biais de mécanismes standard, qui permettent une utilisation depuis de multiples types de terminaux (par exemple, les ordinateur portables, tablettes, smartphones) (Mell, 2011). 
   \item \textbf{Mutualisation de ressources} (Pooling). Les ressources du cloud peuvent être regroupées pour servir des utilisateurs multiples, pour lesquels des ressources physiques et virtuelles sont automatiquement attribuées (Mell, 2011). En général, les utilisateurs n’ont aucun contrôle ou connaissance sur l’emplacement exact des ressources fournies. Toutefois, ils peuvent imposer de spécifier l’emplacement à un niveau d’abstraction plus haut.
   \item \textbf{Scalabilité et élasticité.} Des ressources supplémentaires peuvent être automatiquement mises à disposition des utilisateurs en cas d’accroissement de la demande (en réponse à l'augmentation des charges des applications) (Geelan, 2008), et peuvent être libérées lorsqu’elles ne sont plus nécessaires. L’utilisateur a l’illusion d’avoir accès à des ressources illimitées à n'importe quel moment, bien que le fournisseur en définisse généralement un seuil (par exemple : 20 instances par zone est le maximum possible pour Amazon EC2).
   \item \textbf{Autonome.} Le cloud computing  est un système autonome et géré de façon transparente pour les utilisateurs. Le matériel, le logiciel et les données au sein du cloud peuvent être 
    	automatiquement reconfigurés, orchestrés et consolidés en une seule image qui sera fournie à l’utilisateur (Wang, 2008).
    	\item \textbf{Paiement à l’usage.} La consommation des ressources dans le cloud s’adapte au plus près aux besoins de l’utilisateur. Le fournisseur est capable de mesurer de façon précise la consommation (en durée et en quantité) des différents services (CPU, stockage, bande passante,…) ; cela lui permettra de facturer l’utilisateur selon sa réelle consommation (Armbrust, 2009). 
    	\item \textbf{Fiabilité et tolérance aux pannes.} Les environnements cloud tirent parti de la redondance intégrée du grand nombre de serveurs qui les composent en permettant des niveaux élevés de disponibilité et de fiabilité pour les applications qui peuvent en bénéficier (Buyya, 2008). 
    	\item \textbf{Garantie QoS.} Les environnements de cloud peuvent garantir la qualité de service pour les utilisateurs, par exemple, la performance du matériel, comme la bande passante du processeur et la taille de la mémoire (Wang, 2008). 
    	\item \textbf{Basé-SLA.} Les clouds sont gérés dynamiquement en fonction des contrats d’accord de niveau de service (SLA) (Buyya, 2008) entre le fournisseur et l’utilisateur. Le SLA définit des politiques, telles que les paramètres de livraison, les niveaux de disponibilité, la maintenabilité, la performance, l'exploitation, ou autres attributs du service, comme la facturation, et même des sanctions en cas de violation du contrat. Le SLA permet de rassurer les utilisateurs dans leur idée de déplacer leurs activités vers le cloud, en fournissant des garanties de QoS. 
    	Après avoir présenté les caractéristiques essentielles d’un service cloud, nous présentons, brièvement, dans la section suivante, quelques technologies connexes aux clouds.
    	
    \end{itemize} 
\subsubsection{Technologies connexes }

\subsection{Modèles du cloud computing  }
\subsubsection {Modèles de service du cloudcomputing}  
XaaS (X as a Service) représente la base du paradigme du cloud computing, où X représente un service tel qu’un logiciel, une plateforme, une infrastructure, un Business Process, etc. Nous présentons, dans cette section,  quatre  modèles de services (Rimal, 2009), à savoir: (1) Logiciel en tant que services SaaS (Software as a Service), où le matériel, l’hébergement, le framework d’application et le logiciel sont dématérialisés, (2) Plateforme en tant que service PaaS (Platform as a Service), où le matériel, l’hébergement et le framework d’application sont dématérialisés, (3) Infrastructure en tant que service IaaS (Infrastructure as a Service) et (4) Matériel en tant que service HaaS (Hardware as a Service), où seul le matériel (serveurs) est dématérialisé dans ces deux derniers cas. La figure \ref{fig:capture5} montre le modèle classique et les différents modèles de service de cloud
\begin{figure}[h]
	\centering
	\includegraphics[width=0.7\linewidth]{Capture5}
	\caption{Les services XaaS du cloud computing}
	\label{fig:capture5}
\end{figure}

\begin{enumerate}
\item 	\textbf{Software as a Service (SaaS):}

Ce modèle de service est caractérisé par l’utilisation d’une application partagée qui fonctionne sur une infrastructure Cloud. L’utilisateur accède à l’application par le réseau au travers de divers types de terminaux (souvent via un navigateur web). L’administrateur de l’application ne gère pas et ne contrôle pas l’infrastructure sous-jacente (réseaux, serveurs, applications, stockage).  Il ne contrôle pas les fonctions de l’application à l’exception d’un paramétrage de quelques fonctions utilisateurs limitées. On prend comme exemple les logiciels de messagerie au travers d’un navigateur comme Gmail ou Yahoo mail. 

\item  \textbf{ Platform as a Service (PaaS):}

L’utilisateur a la possibilité de créer et de déployer sur une infrastructure Cloud PaaS ses propres applications en utilisant les langages et les outils du fournisseur. L’utilisateur ne gère pas ou ne contrôle pas l’infrastructure Cloud sous-jacente (réseaux, serveurs, stockage) mais l’utilisateur contrôle l’application déployée et sa configuration. Comme exemple de PaaS, on peut citer un des plus anciens -IntuitQuickbase- qui permet de déployer ses applications bases de données en ligne ou -Google Apps Engine (GAE)- pour déployer des services Web. 

Dans ces deux cas l’utilisateur de ces services n’a pas à gérer des serveurs ou des systèmes pour déployer ses applications en ligne et dimensionner des ressources adaptées au trafic.
\item   \textbf{Infrastructure as a Service (IaaS):}

L’utilisateur loue des moyens de calcul et de stockage, des capacités réseau et d’autres ressources indispensables (partage de charge, pare-feu, cache). L’utilisateur a la possibilité de déployer n’importe quel type de logiciel incluant les systèmes d’exploitation. L’utilisateur ne gère pas ou ne contrôle pas l’infrastructure Cloud sous-jacente mais il a le contrôle sur les systèmes d’exploitation, le stockage et les applications. Il peut aussi choisir les caractéristiques principales des équipements réseau comme le partage de charge, les pare-feu, etc. L’exemple emblématique de ce type de service est Amazon Web Services qui fournit du calcul (EC2), du stockage (S3, EBS), des bases de données en ligne (SimpleDB) et quantité d’autres services de base. Il est maintenant imité par de très nombreux fournisseurs.
	\item \textbf{Points fortset Points faibles des services cloud:} 
\end{enumerate}
\subsubsection { Modèles de déploiement}  

Selon la définition du cloud computing  donnée part le NIST (Mell, 2011), il existe quatre modèles de déploiement des services de cloud, à savoir : cloud privé, cloud communautaire, cloud public et cloud hybride, comme illustré dans la figure \ref{fig:cloudmd}.
\begin{enumerate}
	\item \textbf{Cloud privé :}\\
	 L’ensemble des ressources d’un cloud privé est exclusivement mis à disposition d’une entreprise ou organisation unique. Le cloud privé peut être géré par l’entreprise ellemême (cloud privé interne) ou par une tierce partie (cloud privé externe). Les ressources d’un cloud privé se trouvent généralement dans les locaux de l’entreprise ou bien chez un fournisseur de services. Dans ce dernier cas, l’infrastructure est entièrement dédiée à l’entreprise et y est accessible via un réseau sécurisé (de type VPN).  L’utilisation d’un cloud privé permet de 	garantir, par exemple, que les ressources matérielles allouées ne seront jamais partagées par deux clients différents. 
	
	\item \textbf{Cloud communautaire:}\\
	 L’infrastructure d’un cloud communautaire est partagée par plusieurs organisations indépendantes ayant des intérêts communs. L’infrastructure peut être gérée par les organisations membres ou par un tiers. L’infrastructure peut être située, soit au sein des dites organisations, soit chez un fournisseur de services. 
	\item \textbf{Cloud public:}\\
	L’infrastructure d’un cloud public est accessible à un large public et appartient à un fournisseur de services. Ce dernier facture les utilisateurs selon la consommation et garantit la disponibilité des services via des contrats SLA. 
	\item \textbf{Cloud hybride:}\\
	 L’infrastructure d’un cloud hybride est une composition de plusieurs clouds (privé, communautaire ou public). Les différents clouds composant l’infrastructure restent des entités uniques, mais sont reliés par une technologie standard ou propriétaire permettant ainsi la portabilité des données ou des applications déployées sur les différents clouds.  

\begin{figure}[h]
	\centering
	\includegraphics[width=0.5\linewidth]{Cloud_MD}
	\caption{Modèles de déploiement du cloud computing }
	\label{fig:cloudmd}
\end{figure}
\end{enumerate}

\subsection{Cloud computing et sécurité
}
Toutes les enquêtes montrent que la sécurité est la préoccupation majeure des organisations dans le processus d’adoption des technologies Cloud. Les questions sont nombreuses comme par exemple :

\begin{itemize}

\item Quelle confiance peut-on avoir dans le stockage des données à l’extérieur de l’entreprise ?
\item Quels sont les risques associés à l’utilisation de services partagés ?
\item Comment démontrer la conformité des systèmes à des normes d’exploitation ?
\end{itemize}

Les infrastructures Cloud sont de gigantesques systèmes complexes. Ils peuvent cependant être réduits à un petit nombre de primitives simples qui sont instanciées des milliers de fois et à quelques fonctions communes. La sécurité du Cloud est donc un problème gouvernable moins complexe qu’il n’y parait.

\subsubsection{Avantages et défis du Cloud en terme de sécurité}

Le Cloud présente des avantages immédiats. D’une manière générale, le fait d’héberger des données publiques sur le Cloud réduit les risques pour les données internes sensibles. D’autre part, l’homogénéité dans la construction du Cloud en rend les tests et les audits plus simples. De même la conduite du système au travers de web services permet la mise en place  de procédures automatiques accroissant notablement la sécurité.

En revanche les défis restent nombreux pour les fournisseurs.  Il faut donner confiance dans le modèle de sécurité et dans les outils de gestion qui sont proposés. Les tâches de gestion sont réalisées de manière indirecte au travers d’une interface puisque l’utilisateur n’a pas de contrôle direct sur l’infrastructure physique. Ce partage des responsabilités complique un peu les audits de sécurité.

\subsubsection{Les composants sécurité d’un système de Cloud computing}
Les différents composants qui participent à la sécurité d’un système de Cloud computing présentent les caractéristiques suivantes :
\begin{itemize}
	 
\item \textbf{Service de console de gestion (Provisioning):}\\
La mise en route et la reconfiguration des composants des systèmes sont très rapides. Il est possible de mettre en service plusieurs instances dans plusieurs centres de traitement répartis dans le monde en quelques minutes. Les reconfigurations réseau sont facilitées. En revanche, la sécurité d’utilisation de la console de gestion devient impérative (authentification multi-facteurs, connexion chiffrée, etc..)

\item \textbf{Service de stockage des données:}\\
Les avantages du stockage des données dans le Cloud dépendent des fournisseurs mais en général, ceux-ci fragmentent et répartissent les données. Celles ci sont aussi souvent recopiées dans des centres de traitement différents. Ces opérations améliorent considérablement la sécurité des données. Si leur contenu doit rester confidentiel, il convient de les chiffrer avant de les stocker.

\item \textbf{Infrastructures de calcul:}\\
Un des gros avantages du Cloud pour le développement et l’exploitation des applications réside dans la virtualisation. Elle permet de préparer des configurations maîtres sûres qu’il suffit de dupliquer pour déployer. Les défis restent la sécurisation des données dans les applications partagées et  la sécurité entre les instances garantie par les hyperviseurs.
\item \textbf{Services de support:}\\
La principale caractéristique du Cloud est la mise en place a priori d’une sécurité renforcée et auditable (authentification, logs, pare-feux, etc..). Il reste à traiter les risques liés à l’intégration avec les applications des utilisateurs ainsi que les processus toujours délicats de mises à jour
\item \textbf{Sécurité périmétrique du réseau Cloud:}\\
Ces grandes infrastructures partagées fournissent des moyens de protection au delà des capacités  d’une entreprise normale comme par exemple la protection contre les attaques DDOS (Distributed Denial Of Service). Les mécanismes de sécurité périmétriques sont généralement bien conçus (fournisseur d’identité, authentification, pare-feux , etc..). En revanche, il reste à traiter les sujets liés à la mobilité.

\end{itemize}

%%%%%%%%%%%%%%%%%%%%%%%%%%%%              Workflow et systèmes de gestion de workflows 
%%%%%%%%%%%%%%%%%%%%%%%%%%%%

 \section{Workflow}
 	 
 	 \subsection{Introduction au Workflow }
 	 
 	 
 	 Un Workflow est la modélisation et la gestion assistée par ordinateur de l’accomplissement des tâches composant un processus administratif ou industriel, en interaction avec divers acteurs (humains, logiciels, ou matériels) invoqués [COURTOIS 96]. Outil informatique d’origine industrielle, le Workflow est l’adaptation de la GED24 adjoint de la faculté à gérer l’échange de messages. Le Workflow propose des solutions d’optimisation et de rationalisation des flux d’informations ; que ces informations soient associées à des documents, des procédures ou des messages complémentant les systèmes de gestion électronique de documents et d’informations.
 	 
 	 A l’heure actuelle plus de 250 Systèmes de Gestion de Workflow (WFMS) sont utilisés ou en développement. Cela signifie que le terme « gestion de Workflow » n’est pas simplement une nouvelle expression à la mode. Ce phénomène de gestion de processus (Workflow) aura certainement un fort impact sur la génération suivante de systèmes informatiques [COURTOIS 96, HAYES 91, KOULOPOULOS 95, SCHAEL 97].
 	 
 \subsection{Origines} 
 	 Il est intéressant de considérer l’évolution des systèmes informatiques au cours des quatre dernières décennies [VAN DER AALST 02] pour prendre conscience de la pertinence d’une gestion électronique de processus (Workflow) et apprécier l’impact de la gestion de Workflow dans un avenir proche.
 	 
 	  La Figure \ref{fig:wfmchistory} présente le phénomène de gestion de Workflow dans une perspective historique. Cette figure décrit l’architecture d’un système informatique classique en termes de composants. Dans les années soixante, un système informatique était composé d’un certain nombre d’applications autonomes. Pour chacune de ces applications une interface utilisateur et un système de base de données spécifique étaient développés, chaque application possédait donc ses propres routines pour interagir avec l’utilisateur, stocker et récupérer les données. Dans les années soixante-dix, le développement des systèmes de gestion de base de données (SGBD) a permis d’extraire les données des applications. En utilisant les SGBD, les applications ont ainsi été libérées du fardeau de la gestion de données. Dans les années quatrevingts, l’apparition de systèmes de gestion d’interface utilisateur « User Interface Management Systems » (UIMS) a permis aux développeurs d’application d’extraire l’interaction avec les utilisateurs des applications. Enfin, les années quatre-vingt-dix sont marquées par l’apparition de logiciels de Workflow, permettant aux développeurs d’application d’extraire les procédures de travail des applications. La Figure \ref{fig:wfmchistory} fait apparaître le système de gestion de Workflow comme une composante générique pour représenter et manipuler les processus d’entreprise25. 
 	  
 	  Ainsi, à l’heure actuelle, beaucoup d’organisations commencent à considérer l’utilité d’outils avancés pour soutenir la conception et l’exécution de leurs processus d’entreprise.
 	 
 	 
 	 
\begin{figure}[h]
	\centering
	\includegraphics[width=0.7\linewidth]{images/wfmcHistory}
	\caption{Les systèmes de gestion de Workflow dans une perspective historique }
	\label{fig:wfmchistory}
\end{figure}
 	 
 	 
 	 \subsection{Définitions et terminologies }
 	 
 	 Les définitions sont, pour la majorité, issues de la Coalition de Gestion de Workflow « Workflow Management Coalition » (WfMC). La WfMC a été fondée en 1993 par un regroupement d’industriels de l’informatique, de chercheurs et d’utilisateurs, associée à l’essor du développement des Workflows. Cette coalition a pour but de promouvoir les Workflow et d’établir des standards pour les « Workflow Management System » (WfMS). Elle a en particulier publié un glossaire de référence contenant les terminologies employées dans ce domaine [WFMC03 95 ; WFMC11 99]. Ces standards servent notamment à résoudre les problèmes d’interopérabilité entre systèmes Workflow mais également à définir les caractéristiques fondamentales de ces systèmes. Les documents publiés par la WfMC, qui couvrent plusieurs aspects, peuvent être considérés comme des références en la matière. 
 	
 	\subsection{Définitions de base du Workflow } 
 	 
 	 Le sens du mot Workflow peut varier en fonction du contexte. Pour plus de clarté, les définitions les plus communément admises sur les concepts et les termes du Workflow sont rappelées ci dessous. Ces définitions sont principalement issues du « Workflow Management Coalition Terminology and Glossary » WFMC-TC-1011 [WFMC11 99], dont il existe une traduction à usage francophone [WFMC03f 98]. L’idée première du Workflow est donc de séparer les processus, les ressources et les applications, afin de se recentrer sur la logistique des processus travail et non pas sur le contenu des tâches individuelles. Un Workflow est donc le lien entre ces trois domaines comme précise la Figure 13. 
 	 
 	 
 	 
 	 
 	 
 	 
 	 
 	 
 	 
 	 
 	 
 	 \subsubsection{Définition d’un Workflow }
 	 
 	 
 	 Le Workflow est une technologie informatique ayant pour objectif la gestion des processus d’organisations ou d’entreprises : les termes suivants sont également employés pour qualifier cette technologie « Système de Gestion Electronique de Processus », « Gestion de Workflow » ou « Gestion de processus » [COURTOIS 96]. 
 	 
 	 Le Workflow est l’ensemble des moyens mis en œuvre pour automatiser et gérer les processus d’une organisation. Cette gestion est rendue possible par la représentation sous forme d’un modèle, de tout ou partie des processus considérés. Le Workflow doit ensuite transcrire les modèles obtenus en une forme exécutable. Enfin, ces modèles sont exécutés et gérés. Il est ainsi possible de suivre l’évolution de leur état au fil du temps. La gestion de processus inclut également, au cours de l’exécution, la coordination et la synchronisation des différents acteurs des processus en fonction de l’état actuel des modèles.
 	 
 	 Pour résumer, la Gestion de Processus permet donc d’attribuer à chacun et au bon moment, les tâches dont il a la responsabilité et de mettre à disposition les applications, les outils et les informations nécessaires pour leurs réalisations. Dans un contexte d’acteurs humains, le Workflow permet de décharger les acteurs de certaines tâches de gestion administrative, en leur laissant la possibilité de se concentrer sur les contenus des tâches techniques en rapport avec leurs compétences. De plus, le Workflow donne la possibilité d’effectuer une activité de monitoring sur le déroulement des Workflow de l’entreprise, permettant en particulier de connaître, en fonction de la date, l’état des activités, des acteurs, des applications et quelles sont les prochaines activités planifiées. 
 	 
 	 En synthèse, La WfMC présente le Workflow comme l’automatisation d’un processus d’entreprise, en intégralité ou en partie, pendant laquelle on définit les transmissions des documents, de l’information ou des tâches d’un participant à un autre pour agir, selon un jeu de règles procédurales [WFMC11 99]. Un Système Workflow définit, gère et exécute des procédures en exécutant des programmes dont l’ordre d’exécution est prédéfini dans une représentation informatique de la logique de ces procédures - les Workflow [WFMC11 99]. 
 	 
 	 \subsubsection{ Méta Modèle basique }
 	 Le Workflow est basé sur un ensemble de concepts. La WfMC [WFMC11 99] a proposé un méta modèle de Définition de Procédures, qui identifie les concepts de haut niveau dans la Définition de Processus. Ce modèle permet de mieux appréhender les concepts et leurs interrelations. 
 	 
 	 
 	 
 	 
\begin{figure}[h]
	\centering
	\includegraphics[width=0.7\linewidth]{images/MetaMWFMC11-99}
	\caption{Méta modèle Workflow pour la définition de Processus [WFMC11 99]}
	\label{fig:metamwfmc11-99}
\end{figure}
 	 
 	 Le méta modèle, présenté Figure \ref{fig:metamwfmc11-99}, identifie un ensemble d’objets fondamentaux qui entrent dans la définition d’un processus géré par un système Workflow, et que nous allons définir et commenter dans les paragraphes suivants. Remarquons que le méta modèle peut être enrichi par les développeurs de systèmes, il peut également être utilisé à des fins d’échanges entre différents systèmes Workflow.
 	 
 	\subsection{Concepts et Terminologie Workflow fondamentaux } 
 	 
 	 Les principaux termes associés aux Workflow proposés par la WfMC [WFMC11 99] sont présentés dans le diagramme du méta modèle Workflow ci-dessus, ce diagramme permet également de mettre en évidence leurs interrelations. Les termes présentés ci-dessous en français avec la traduction anglaise originale associée, couvrent les notions plus importantes appartenant au Workflow et à son lexique [WFMC11 99]. 
 	 
 	 \subsubsection{Procédure Workflow (Workflow Process) }
 	 
 	 Une procédure Workflow est une procédure contrôlée par un Workflow. Une procédure est composée de plusieurs activités enchaînées pour représenter un flux de travail. Une procédure possède une structure hiérarchique et modulaire, en l’occurrence une procédure peut donc être composée de sous procédures et d’activités. Les sous-procédures peuvent être composée elles mêmes de procédures manuelles ou de procédures Workflow.
 	 
 	 
 	 \subsubsection{Activité (Process Activity) }
 	 Une activité est une étape d’un processus au cours de laquelle une action élémentaire est exécutée. On désigne par « action élémentaire » (ou tâche) une activité qui n’est plus décomposable en sous-procédures. La WfMC distingue une « activité manuelle », qui n’est pas contrôlée par le système Workflow, et une « activité Workflow » qui est sous le contrôle du Workflow. Un exemple d’une activité manuelle est l’ouverture d’un courrier. Une activité Workflow peut être le remplissage d’un formulaire électronique. Il existe donc des exemples d’activités manuelles intégrables dans un Workflow. 
 	 
 	 [VAN DER AALST 98a] présente l’activité Workflow comme l’intersection entre une ressource humaine ou matérielle et un bon de travail dans le cadre de l’exécution d’une tâche. Dans cette représentation, une ressource du modèle organisationnel est donc exigée pour qu’une tâche puisse être instanciée en activité et allouée à un participant de Workflow. 
 	 
 	 
 	 
 	  	 \subsubsection{ Acteur, Ressource (Workflow Participant) }
 	 
 	Un acteur est une entité du modèle organisationnel participant à l’accomplissement d’une procédure. L’acteur est chargé de réaliser les activités qui lui sont attribuées via le(s) rôle(s) qui lui sont définis dans le modèle organisationnel. Les autres dénominations courantes dans la littérature de cette entité sont « ressource », « agent », « participant » ou « utilisateur ». L’acteur peut être une ressource humaine ou matérielle (machine, périphérique informatique…).
 	 
 	 Les ressources sont organisées en classes dans le modèle organisationnel. Ces classes sont des groupes de ressources possédant des propriétés communes. Une classe est basée sur :
 	 
 	  Rôle : défini ci dans le § suivant. 
 	  
 	 Groupe : cette classification est basée sur l’organisation (département, équipe, unité). 
 	 
 	 \subsubsection{ Rôle (Role) }
 	 
 	 
 	 Un rôle décrit en général les compétences d’un acteur dans le processus ou sa position dans l’organisation. Un rôle est associé à la réalisation d’une ou de plusieurs activités. Plusieurs acteurs peuvent tenir un même rôle. La WfMC distingue deux types de rôles [WFMC11 99] :
 	 
 	 Les rôles organisationnels définissent un ensemble de compétences qu’un acteur possède. Ce rôle définit la position de l’acteur dans une organisation. Les rôles procéduraux définissent une liste d’activités qu’un acteur est en capacité d’exécuter. 
 	 
 	 Il est à noter que certains travaux ne différencient pas les notions d’acteur et de rôle et ne parlent que d’acteur. Cette opinion semble restreindre la clarté et la flexibilité des modèles Workflow. 
 	 
 	 	 \subsubsection{Données (Workflow Relevant Data) }
 	 
 	 Une donnée pertinente pour les procédures est une information en rapport avec la réalisation des activités (en définition de la tâche, en entrée ou en sortie). Elle peut constituer l’objectif d’une tâche (manipulation de la donnée et définition de l’état de la procédure), être un élément essentiel pour activer les transitions d’état d’une instance Workflow ou être généré par la tâche et ainsi intervenir dans la détermination de la prochaine activité à déclencher. Ces données sont en général des objets au sens purement informatique mais peuvent également être une représentation d’objets physiques. 
 	 
 	 Notons qu’il existe deux autres types de données utilisées hors de la gestion de procédures : 
 	 
 	 Donnée de contrôle (Control Data) : données gérées et utilisées par le système Workflow et les moteurs Workflow.
 	 
 	 Données Applicatives (Applicative Data) : données propres aux applications, le système de gestion de Workflow n’y a pas accès. 
 	 
 	 \subsubsection{ Application externe (Invoked Application) }
 	 
 	 Une application externe est une application informatique dont l’invocation est nécessaire à la réalisation de la tâche ou à l’exploitation des résultats générés avant de déclencher la tâche suivante ou de recommencer cette première. On tiendra compte de l’allocation de ressources, si l’application n’est pas uniquement informatique. Il faut différencier les outils (Tools), qui sont eux directement interfacés par le système Workflow, sans l’intervention d’une ressource du Système Workflow.
 	 
 	 
 	 
 	 
 	 
 	 
 	 
 	 
 	 \subsection{ Concepts de base et définitions de Workflow }
 	 
 	
 
La notion de workflow (traduit en français par "flux de travail") est apparue dans l’industrie de l’image électronique et de la gestion de production assistée par ordinateur (GW, 1998). Ce concept a donc été créé dans le but d’automatiser les procédures de travail au sein des organisations. L’idée d’enchaîner différentes tâches pour réaliser un traitement complexe est pertinente. De plus, dans les infrastructures actuelles distribuées, gérant des ressources hétérogènes, telles que le cloud computing, bénéficier d’un environnement autorisant la définition et l’exécution des chaînes de traitement constitue une des fonctionnalités essentielles recherchée, à la fois par les scientifiques et au-delà par le grand public.

Deux grandes catégories d’usages utilisent la notion de workflow : les protocoles expérimentaux, dans des domaines tels que la biologie, l’astronomie, la physique, la neuroscience, la chimie, etc. (workflows scientifiques) et les chaînes de traitement pratiquées dans des domaines commerciaux, financiers, pharmaceutiques (processus métiers). Elles donnent lieu à plusieurs pistes de recherche diverses, mais cependant connexes. Dans le cadre de cette thèse, nous traitons plus particulièrement les workflows scientifiques.

\subsubsection{Définitions  de base du  workflow}


Avant de définir le terme workflow, il est à noter qu'un problème de confusion persiste entre les termes : workflow (processus workflow), technologie workflow et système workflow. En ce qui suit, nous allons définir chacun de ces termes.

\begin{enumerate}
	\item \textbf{Définition 1: }  Un\textbf{ workflow} est la forme exécutable d'un processus d'une organisation, gérable par un système workflow. Il permet d'automatiser l'exécution du processus ou encore sa simulation.
	
	\item \textbf{Définition 2: }Un \textbf{système workflow} (ou \textbf{WfMS} pour Système de Gestion de Workflow) est un système informatique permettant la gestion des processus métiers. Les services proposés par un WfMS sont au minimum l'exécution d'un processus et sa gestion (contrôle et suivi) en plus de la mise à disposition des outils et des documents nécessaires à la réalisation des différentes étapes du processus. 
	\item \textbf{Définition 3: }\textbf{La technologie workflow} est la technologie informatique du \textbf{TCAO} (Travail Coopératif Assisté par Ordinateur), qui s'intéresse à la gestion des processus de l'organisation. C'est l'ensemble des moyens utilisés pour automatiser et gérer un processus. Cette gestion est garantie vu qu'il est possible de présenter un modèle de processus sous une forme exécutable.
\end{enumerate}

La relation entre ces trois termes est la suivante : Une entreprise peut introduire une technologie workflow dans son système d'information en installant un système de gestion de workflow qui gère ses processus, automatisés en workflow.

La WfMC (Workflow Management Coalition) (WfMC, 1999) a donné une définition qui généralise la notion de workflow indépendamment des domaines spécifiques:

\textit{"Workflow is the automation of business process, in whole or part during which documents, information or tasks are passed from one participant to another for action, according to a set of procedural rules."}

Nous traduisons cette définition par: " Un workflow est l'automatisation d'un processus métier, en tout ou en partie, au cours de laquelle des documents, des informations ou des tâches sont passées d'un participant à un autre pour l'action, selon un ensemble de règles procédurales". 

En ce qui concerne le workflow scientifique, nous retiendrons la définition suivante. 

Un workflow est composé d’un ensemble de tâches (traitements) organisées selon un ordre logique, afin de réaliser un traitement global, complexe et pertinent sur un ensemble de données sources. Ces données sont souvent complexes, tant au niveau de leurs structures que de leurs organisations. Elles sont souvent volumineuses.

La taille d’un workflow scientifique peut varier de quelques tâches à des millions de tâches, qui sont souvent de calcul intensif "Computation Intensive" (Ludäscher, 2009).  Pour les grands workflows, il est souhaitable de répartir les tâches entre plusieurs ressources, afin d’optimiser les temps d’exécution. En tant que tel, les workflows impliquent souvent  des calculs répartis sur des clusters, des grilles, et d'autres infrastructures informatiques. Récemment, les clouds computing  sont évalués comme une plateforme d'exécution de workflows (Hoffa, 2008; Juve, 2008). L’exécution d’un workflow est gérée par un SGWf (Système de Gestion de Workflow) dont l’architecture de référence est décrite brièvement dans la section suivante. 
1

\subsubsection{Concepts et terminologie de workflow}
  La liste suivante présente les concepts de base de workflow et les structures de base pour la conception de workflow et le contrôle de processus comme le suggère la WfMC \parencite{WFMC}:
  
 \begin{itemize}
 \item Une \textbf{activité} (tâche): est une description d'une partie du travail qui constitue une étape logique dans un workflow. Elle peut être manuelle ou automatique [1]. Une activité manuelle est entièrement réalisée par une ou plusieurs personnes, sans aucune utilisation d'une application. En revanche, une activité automatique est effectuée par une application, sans aucune intervention des personnes, en se basant sur des données déjà enregistrées. Les activités sont classées en fonction des mutuelles dépendances imposées par des aspects structurels et de données (flot de contrôle et flot de données entre les activités). Différentes configurations permettent de couvrir les aspects structurels : séquence, sélection, itération, et concurrence. Pour la représentation des données, deux approches sont les plus utilisées : soit par le biais des flux de données entre les activités, soit par l'intermédiaire des services de fourniture des données des (ou vers les) activités .
\item Une \textbf{instance }(instance de workflow (un cas) ou instance d'activité): est la représentation d'une exécution unique d'un workflow ou d'une activité dans un workflow.

\item Un \textbf{ participant} (acteur, agent, utilisateur, entité de traitement, ressource): est une entité qui exécute une instance d'activité. Cette entité peut être un être humain ou un système logiciel.
 \item Un \textbf{élément de travail} (work-item): est la représentation du travail à traiter (par un participant) dans le cadre d'une activité d'une instance de workflow. Une liste des éléments de travail associée avec un participant de workflow donné (ou groupe de participants) est appelé une liste de travail (work-list).
 \item  Un \textbf{état de workflow} (resp. d'activité): est lié à des conditions internes déffnissant l'état d'une instance du workflow (resp. de l'activité) à un moment donné. Dans le cas d'un workflow, l'état pourrait être "initié", "en exécution", "actif", "suspendu", "achevé", "terminé" et "archivé". Dans le cas d'une activité, il pourrait être "inactive", "active", "en exécution", "suspendue", "sautée" et "terminée". 
 \end{itemize}
En résumé, nous distinguons dans un workflow des cas, des éléments de travail (workitems) et des ressources. Les work-items lient les cas et les tâches, les activités lient les cas, les tâches et les ressources.
La figure  [7] montre qu'un workflow comporte trois dimensions : (1) la dimension de cas, (2) la dimension du processus et (3) la dimension des ressources. La dimension de cas signifie le fait que tous les cas sont traités individuellement. Du point de vue workflow, les cas ne s'influencent pas des autres, mais ils s'influencent les uns des autres indirectement via le partage des ressources et des données. Dans la dimension de processus, il est spécifeé le processus de workflow, c'est à dire les tâches et l'acheminement de ces tâches. Dans la dimension des ressources, les ressources sont regroupées dans des classes particulières nommées les rôles et les unités organisationnelles. Une classe de ressource est un ensemble de ressources présentant des caractéristiques similaires. Si une classe de ressource est basée sur les capacités (exigences fonctionnelles) de ses membres, elle est appelée un rôle. Si le classement est basé sur la structure de l'organisation, une classe de ressource est appelée une unité organisationnelle (par exemple une équipe ou un département). 

\section{Classification des systèmes Workflow}

Il n’existe pas de classification commune des systèmes Workflow dans la littérature, reconnue par l’ensemble de la communauté Workflow [VAN DER AALST 02]. Ceci étant essentiellement dû au nombre important de critères de classification qu’il est possible de retenir.
En effet, les spécialistes adoptent différents points de vue par rapport à la notion de Workflow, les critères qui en découlent varient donc en fonction de leurs perceptions des caractéristiques présentées par ces systèmes Workflow. Ainsi, il existe plusieurs classifications, permettant de sélectionner un outil de gestion de Workflow avec différents « éclairages » sur le sujet.

Malgré ce manque d’unité, la classification proposée par [McCREADY 92] est assez répandue dans la littérature, elle est reprise par bon nombre d’auteurs [VAN DER AALST 98a],
[GEORGAKOPOULOS 95]. Elle propose de distinguer quatre catégories de Systèmes Workflow. La Figure fig:classification-de-workflow présente ces différentes classes selon deux axes : Approche et Structure.

\begin{figure}[h]
	\centering
	\includegraphics[width=0.7\linewidth]{"images/classification de workflow"}
	\caption{ Différentes classes des systèmes Workflow 
}
	\label{fig:classification-de-workflow}
\end{figure}


\subsection{Processus collaboratifs }
Cette première classe est axée sur la communication et sur le partage d’information. Les systèmes collaboratifs sont définis pour supporter le travail en groupe, dans le cadre de la conception, de la gestion de projet ou de la résolution de problèmes faisant appel à plusieurs niveaux d’expertise. Ces systèmes permettent de réunir les intervenants d’un projet autour d’un objectif commun, les clients de la procédure y étant souvent eux-mêmes directement associés, les logiciels employés sont le plus souvent des groupwares
27. Les tâches des procédures gérées sont le plus souvent complexes et leur réalisation implique l’intervention de ressources aux compétences très spécifiques pour une forte valeur ajoutée. D’un autre coté, l’enchaînement des activités des procédures à traiter est faiblement structuré et peu répétitif.
De part la faible structure de ces processus, ils ne font pas partie ou se situe à la frontière de ce que l’on considère comme la « sphère » Workflow comme le précise la Figure \ref{fig:classification-de-workflow}. 

\subsection{Workflow administratif }
Les systèmes Workflow administratifs (General Purpose Workflow Management Systems) ont pour objectif de décharger les ressources d’une entreprise des tâches administratives. En effet ces procédures sont répétitives, fortement prédictibles et les règles
d’enchaînement des tâches sont très simples et clairement définis ; ces procédures sont donc
aisément automatisables, évitant ainsi un travail fastidieux où peuvent naître des erreurs souvent humaines. Les systèmes Workflow administratifs permettent de lier à une tâche administrative, les documents et les informations nécessaires à la réalisation de cette tâche par un acteur humain. Ces systèmes gèrent également le routage des documents et le remplissage de
formulaires. La gestion par Workflow de procédures administratives permet un gain de
l’ordre de 5\% à 10\% en termes de productivité et de 30 à 90\% en termes de délais [ADER
99]. Enfin une dernière raison de l’automatisation de ce type de procédures en Workflow provient du fait que ces procédures possèdent une structure statique et ne sont donc pas souvent
assujetties à modifications car elles possèdent une longue durée d’utilisation [SCHEER 97]. 
\subsection{Workflow de production} 
Les systèmes Workflow de production impliquent des procédures prévisibles et assez répétitives. Leurs principales différences avec les Workflow administratifs résident dans la
complexité des tâches et de la structure des procédures, dans leur capacité à faire appel à des
informations provenant de systèmes d’information variés et dans l’enjeu que représente leur
réussite. En effet, la procédure Workflow correspond directement au travail effectué par
l’entreprise. En d’autres termes, la performance de l’entreprise est directement liée à
l’exécution de la procédure managée par le Workflow. On dit dans ce cas qu’il est mission
critical28 [INCONCERT 97]. C’est par exemple le cas des organismes financiers, des compagnies d’assurances, des usines de production manufacturières. La réalisation des procédures
est donc associée à une forte valeur ajoutée et un volume d’informations traitées important.
La complexité des procédures traitées est également due à la répartition de leurs activités
sur plusieurs sites. Dans ce cas, les tâches exécutées nécessitent souvent l’interrogation de
plusieurs systèmes informatiques, hétérogènes et distribués. Il est donc nécessaire que les systèmes Workflow de production fournissent un ensemble d’outils ou de fonctions d’API per-mettant de se connecter à plusieurs systèmes. Enfin, même si les procédures traitées sont assez répétitives, elles sont susceptibles d’être modifiés plus souvent que les procédures administratives, car associées à la modification des objectifs du métier. Ces modifications peuvent
par exemple avoir lieu dans le cadre d’une restructuration de BPR29 ou d’un CPI30. 

Les systèmes Workflow de production doivent donc pouvoir évoluer. Par ailleurs,
l’exécution de certaines procédures ne peut pas toujours se poursuivre de manière automatisée, suite à l’occurrence d’un ou de plusieurs événements qui font aboutir le système dans un
état particulier. Dans ce cas, il est nécessaire de faire intervenir des acteurs humains pour la
prise de décision. Pour ce faire, le système Workflow de production peut faire appel à un autre système, de type collecticiel ou un autre système Workflow ad hoc, qui servira d’interface
pour l’exécution dirigée par un acteur humain de la suite de la procédure. Ce type de Workflow est dit « composite » [EDER 96]. Enfin, dans la littérature, les systèmes Workflow de
production sont également appelés case-based [VAN DER AALST 98a]. 

\subsection{Workflow adaptable ou Workflow ad hoc }
L’impossibilité pour les systèmes de gestion de Workflow traditionnels de traiter les différents changements dynamiques dans les flux de travail est une limite à dépasser. A ce titre, il a été introduit les concepts de Workflow adaptable (adaptive Workflow) [VAN DER
AALST 98c] et de Workflow ad hoc [VOORHOEVE 97]. La nuance entre ces deux nouveaux termes provient du fait qu’ad hoc désigne un acte spécialement fait pour un objet déterminé alors qu’adaptable prévoit un changement définitif de la procédure. 

Les Workflow ad hoc se situent à la frontière gauche de la représentation Figure \ref{fig:classification-de-workflow} dans
la « sphère » Workflow adaptable. Ils régissent des procédures dont la structure est déterminée pendant l’exécution en fonction des décisions humaines prises suite à la réalisation d’une
tâche, plus concrètement, la structure se construit par pas en suivant le rythme de l’exécution. En effet, la réalisation d’une procédure non structurée peut impliquer à chaque fois l’exécution d’un nouvel enchaînement des tâches, voire la création de nouvelles tâches. Il n’y a pas a priori de persistance de l’enchaînement de ces tâches.

Les Workflow adaptables sont, quant à eux, des supports comparables aux Workflow de
production classiques possédant une structure préétablie, mais pouvant traiter certains changements de structure « en ligne ». Ces changements peuvent aller des changements individuels/ad hoc (gestion d’exception), c’est à dire d’un aiguillage pour déterminer l’activité suivante, jusqu'à la reconception par BPR de processus [VAN DER AALST 98d]. En conclusion, ils ont une action globale pouvant inclure la définition du Workflow ad hoc  

Il est intéressant de classifier les différents changements possibles par un Workflow adaptables, dans le but de mieux les anticiper [SADIQ 99]. Les changements sont envisageables
selon plusieurs perspectives : la ressource, le contrôle, la procédure, la tâche et le système [VAN DER AALST 98d]. Dans la suite de l’étude seul l’aspect procédure sera développé,
c’est en effet la perspective dominante du management par Workflow et elle comporte un
aspect important : les changements dynamiques. Nous présentons ci-dessous les différents
types de changements envisageables. 

\subsubsection{Le changement individuel (ad hoc) }
Les systèmes Workflow ad hoc sont utilisés pour l’exécution de processus non structurés
ou peu structurés (sujets à changement). Un processus peu ou non structuré est un processus
dont l’ordre et le temps exact de réalisation des tâches ne sont pas établis au préalable et/ou peuvent être modifiés pendant l’exécution. Les choix de routage et la nature des tâches sont décidés au fur et à mesure de l’exécution. Par conséquent, un processus non structuré propose un objectif immuable, mais pouvant être atteint de différentes façons. Certaines situations rencontrées pendant le « run time » nécessitent donc des dérivations ad hoc dans la procédure, éventuellement planifiée, comme le proposent [HAN 98], telles que :
\begin{enumerate}

\item \textbf{Le raffinement Dynamique : }\\
Dans certains cas, il est impossible ou peu pratique de définir une spécification complète
du modèle de Workflow. En raison de l’indisponibilité d’une spécification complète, le raffinement dynamique peut être nécessaire pendant le « run time », c’est-à-dire que certaines tâches ne seront complètement et définitivement spécifiées qu’en « run time ». Le même raisonnement peut être appliqué pour définir les ressources exigées pour l’exécution d’une tâche. 

\item \textbf{La participation d’Utilisateurs : }\\
Au lieu d’être des contrôleurs passifs, certains utilisateurs d’un système de Workflow
doivent être traités comme des « propriétaires d’une tâche ou d’une procédure » [HAN 98].
L’approche de ces systèmes est souvent de type « pull », c’est-à-dire que leurs utilisateurs
doivent les interroger pour connaître l’état du processus et en déduire leurs tâches ; par opposition, les autres types de systèmes, possède eux une approche plutôt de type « push », où les
utilisateurs sont informés par le système des travaux qu’ils ont à traiter [GEORGAKOPULOS
95]. Techniquement, la métaphore utilisée dans ce type de système est celle du « dossier »
[WAINER 95]. Les utilisateurs font circuler un dossier virtuel dans lequel sont « placés » des
documents et des données électroniques. Chaque utilisateur en possession du dossier décide
du prochain destinataire. Le processus décisionnel de l’utilisateur doit être considéré dans
l’exécution de processus de Workflow. 

\item \textbf{Adaptation aux événements externes : }\\
Des événements non pris en compte par le modèle de Workflow, y compris certains stimuli externes, l’intervention d’utilisateurs, les temps morts, etc., doivent être traités correctement pour résoudre les problèmes du monde réel et faciliter la communication entre des procédures Workflow différentes. En outre, une fois qu’une communication inter-Workflow a
lieu, les utilisateurs ou les propriétaires de procédures doivent être capables de répondre à ces événements en raffinant dynamiquement leur procédure ou en modifiant la tâche actuelle
et/ou les interdépendances de tâches. 
\item \textbf{Situation d’échec : }\\
Un défaut système, des conflits de ressource et des fausses opérations peuvent causer des
erreurs et des difficultés dans l’exécution d’une procédure Workflow. Les mécanismes pour
traiter les situations d’erreur sont donc très importants pour assurer l’amélioration des processus de Workflow. 

\end{enumerate}
\subsubsection{Le changement structurel (évolution) }
Ces changements sont souvent une réaction pour s’adapter à un changement
d’environnement dû au contexte concurrentiel très dynamique ou au besoin d’adaptation aux
progrès technologiques, au travers la parution de nouveaux logiciels ou de nouvelles versions
[HAN 98].

Ces changements sont souvent le fruit d’un travail de BPR.

Après une telle modification, il existe plusieurs possibilités [VAN DER AALST 98d] et [SADIQ 99] d’intégrer les cas existants dans la nouvelle procédure, contrairement aux changements ad hoc qui restent un traitement d’exceptions et sont gérés individuellement.

Première possibilité : Redémarrer « restart » « abort »

Les cas en cours de traitements dans l’ancien processus sont remis à zéro et redémarrés
dans le nouveau processus au lancement du nouveau système.

Deuxième possibilité : Parallèle « proceed » « flush »

Le système conserve en parallèle l’ancien et le nouveau processus le temps de l’exécution
des cas en cours sur l’ancien processus.

Troisième possibilité : Transférer « transfer » « migrate »

Ce changement n’affecte pas le traitement des cas qui sont transférés directement dans le
nouveau processus dans l’état actuel de leur déroulement. 









 



\subsection{Comparaison entre types de workflows:} \ref{tab:tabl2}
 
\begin{center}
\begin{table}[h]
	\centering
	
	\begin{tabular}{| m{2cm} |m{8em}| m{8em} |m{6em}|m{7em}|}
		\hline
	\rowcolor[HTML]{38FFF8} 
		\textbf{Critères} & \textbf{De production} & \textbf{Administratif} & \textbf{Ad-hoc} & \textbf{Collaboratif} \\ \hline
		
		\textbf{Capacité de traitement} &   Haute capacité de traitement Temps de réponse rapide.Le but est la Productivité & Capacité de traitement inferieure(10 à 100)fois moins que pour un workflow de production &  Facilite d'utilisation et d'apprentissage sont très importantes. &   Capacité de changer dynamiquement la définition d’un processus est essentielle \\ \hline
		
	\rowcolor[HTML]{96FFFB} 
		\textbf{Utilisation} &  Employés travaillant à plein temps sur des activités  courtes. &  Un grand nombre d'employés peuvent être   impliqués & La modification dynamique et rapide des  processus est essentielle.  &  Fournir une voie structurée pour travailler ensemble  \\ \hline
		
		\textbf{ Nature des processus } & Processus formels avec peu de variation Les  processus peuvent  être trèscomplexes.&  Une variété de processus pout exister dans même système. Les processus peuvent être bien définis, mais  requièrent moins d'exigence.  & Facilité de  déploiement.&Les processus sont moins rigides \\ \hline
		
	\rowcolor[HTML]{96FFFB} 
		\textbf{Spécificités} &  Requiert une intégration serrée avec les systèmes de bases. &  Utilise souvent des documents attaches.& Le but est de zéro coût  d’administration .&La capacité de traitement est de moindre importance
		\\ \hline
	\end{tabular}
	
	\caption{Comparaison entre types de Workflows.}
	\label{tab:tabl2}
\end{table}
\end{center}
 

\textbf{Exemples de workflows :}
\begin{itemize}
	\item Processus de déclaration de sinistre,
	\item Processus d'ouverture compte,
	\item Processus de création d'un dossier de prêt,
	\item Processus de gestion d'une succession,
	\item Processus de prise de congés.
\end{itemize}
\section{Architecture des systèmes de gestion de workflows }
\subsection{Définition }
La gestion du workflow est une technologie en évolution rapide, qui est de plus en plus exploitée par les entreprises. Un SGWf représente un système qui définit, implémente et gère l'exécution de workflows à l'aide d'un environnement logiciel fonctionnant avec un ou plusieurs moteurs de workflows et capable d'interpréter la définition d'un processus, de gérer la coordination des participants et d'invoquer des applications externes.  

 L'architecture de référence d’un SGWf proposée par la Workflow Management Coalition (WfMC, 95) en1995 est présentée dans la figure \ref{fig:capture6}. Ce modèle inclut un service de déploiement, qui contrôle l'exécution des workflows et qui supporte cinq interfaces standardisées:
 
 \subsection{Modèle de référence des systèmes Workflow }
 
 Le modèle de référence, Figure 18, présente l’architecture générale de l’environnement
 proposée par la WfMC, il identifie les interfaces couvrant cinq domaines de fonctionnalités entre le système Workflow et son environnement. 
 
\begin{figure}[!h]
	\centering
	\includegraphics[width=0.6\linewidth]{Capture6}
	\caption{ Modèle de référence des systèmes de gestion de workflow (WfMC, 95). }
	\label{fig:capture6}
\end{figure}


\subsubsection{Interface avec les Outils de définition de procédures }
Cette interface, située entre les outils de modélisation/définition et le logiciel de gestion
du Workflow pendant l’exécution, est nommée interface d’import/export de définition de processus. Cette interface définit le format d’échange et d’appels des APIs, qui permettent
l'échange d'informations de définition de procédures sur une variété de médias d'échange :
physiques ou électroniques. Cette interface permet l'échange d'une définition de processus
complète ou d’un sous-ensemble. Par exemple le changement de définition d’un ensemble de
procédures ou plus simplement la modification des attributs d'une activité particulière dans
une définition de procédures. 

\subsubsection{Interface avec les applications clientes Workflow }
La liste des tâches (Worklist) à exécuter par une ressource est généralement définie et gérée par le service d’exécution du Workflow. Cette liste doit pouvoir déclencher des appels à
des applications clientes diverses et des ressources. La solution retenue pour respecter la susdite exigence, consiste à encapsuler la variété d’application qui peut être utilisée derrière un jeu standard d'API (le WAPI Workflow Application Programming interface). Ce jeu permet ainsi d’utiliser une communication standardisée entre les applications clientes, le moteur de Workflow et les Worklist, indifféremment de la nature de l’implémentation réelle des produits
clients. 
\subsubsection{Interface avec les applications invoquées}
Il est évident que le système Workflow ne peut pas intégrer l’invocation automatique de
toutes les applications qu’il peut être amené à utiliser pendant l’exécution d’un Workflow. Par
exemple les applications dont les données sont fortement typées. Dans ce cas un composant
externe supplémentaire, nommé agent d’application, est ajouté, il est chargé de la traduction
des informations dans un format compréhensible par le standard WAPI. 

Dans le cas le plus simple, l'invocation d'application est traitée localement par un moteur
de Workflow, mais les applications invoquées peuvent être utilisées par plusieurs moteurs de
Workflow et peuvent se situer sur des machines distantes, il convient donc de définir un format commun d’utilisation des ces applications entre les Workflow dans le but de communiquer correctement et de synchroniser l’appel à ces applications. 

\subsubsection{Interface avec les autres Workflow }
Un des objectifs de la normalisation dans la définition de Workflow est de pouvoir
transmettre des WorkItem entre deux systèmes Workflow conçus par des concepteurs de systèmes Workflow différents. Trois principaux types d’interopérabilité ont étés identifiés : 

\textbf{Workflow chaînés :} La dernière activité d’un Workflow A doit pouvoir fournir un item à la première activité d’un Workflow B.

\textbf{ Workflow hiérarchiques :} une activité d’un Workflow A doit pouvoir être vu comme un
 Workflow B. 
 
 \textbf{Workflow Peer to Peer :} Une procédure globale est composée d’activités gérées en partie par un Workflow et en partie par un autre Workflow, sans système de supervision de
 la procédure complète. 
 
 \textbf{Workflow Synchronisés :} Deux Workflow s’exécutent en parallèle et doivent pouvoir se
 synchroniser sur certaines activités. 
 
 Pour résumer, il est possible d’identifier deux aspects principaux nécessaires à
 l’inter fonctionnement de Workflow :

 • L’interprétation commune de la définition de procédures (ou d’un sous-ensemble).
 
 • L’appui pendant l'exécution de l’échange des divers types d'information de contrôle et
 le transfert des données appropriées et/ou d'applications entre les services d’exécution Workflow différents.
 
 \subsubsection{Interface avec les outils de contrôle et d’administration }


L’objectif de cette interface est de permettre à un logiciel de Monitoring de Workflow de
s’interfacer avec plusieurs Workflow différents et ainsi regrouper la supervision d’un ensemble de systèmes Workflow dans un logiciel.


L'interface 5 permet à une application de gestion indépendante d’interagir avec des
Workflow de différents domaines. L'application de gestion peut aussi se charger d'autres fonctions de gestion, au-delà de celles-ci. Par exemple, elle peut aussi gérer des définitions de procédures de Workflow, agissant comme un dépôt d’information commun à plusieurs systèmes
et distribuant des définitions de processus aux divers Workflow via des opérations au travers
de leurs interfaces 1.
Malgré cela, des scénarii d’implémentations moins modulaires sont aussi envisageables;
par exemple l'application de gestion peut être une partie intégrante du service d’exécution.  
 
 
 
 
\subsubsection{Standards utilisés dans les SGWf :}

\begin{figure}[!h]
	\centering
	\includegraphics[width=0.6\linewidth]{Capture7}
	\caption{Différents standards adoptés dans les SGWf}
	\label{fig:capture7}
\end{figure}


\begin{enumerate}
\item \textbf{ BPNM (Business Process Model and Notation):} est une représentation graphique permettant de spécifier les processus métier maintenus par le groupe de gestion d'objets (OMG). 
\item \textbf{XPDL (Process Definition Language):} est un format normalisé par la WfMC (Workflow Management Coalition) pour l’échange de définitions de partenaire entre différents produits de flux de travail, c.-à-d. entre différents outils de modélisation et suites de gestion. XPDL définit un schéma XML pour spécifier la partie déclarative du workflow / partenaire.
\item \textbf{BPAF (Business Process Analytics):} fournit aux participants aux processus et aux décideurs des informations sur l'efficacité des processus organisationnels.
\item \textbf{BPEL (Business Process Execution Language):} BPEL est un langage d'orchestre.
\item \textbf{ Wf-XML }est un standard BPM développé par la Workflow Management Coalition. Wf-XML offre à un moteur BPM un moyen standard d'appeler un processus dans un autre moteur BPM et d'attendre qu'il se termine.
\end{enumerate}
 

\subsection{ Intérêt du cloud pour les workflows }
Les clouds offrent plusieurs avantages pour les applications à base de workflows. 
Ces avantages facilitent: 
\subsubsection{L’approvisionnement de ressources }
Dans les grilles, l'ordonnancement est basé sur un modèle en best-effort, dans lequel l’utilisateur spécifie la quantité de temps nécessaire et délègue la responsabilité de l'allocation des ressources et d'ordonnancement de tâches à un ordonnanceur fonctionnant en mode batch utilisant des files d’attentes. Dans le cloud, au lieu de déléguer l’allocation au gestionnaire de ressources, l'utilisateur peut provisionner les ressources nécessaires et ordonnancer les tâches en utilisant un ordonnanceur contrôlé par l'utilisateur. Ce modèle d’approvisionnement est idéal pour les workflows, car il permet au système de gestion de workflow d'allouer une ressource une seule fois et de l'utiliser pour exécuter de nombreuses tâches. 
\subsubsection{L’allocation dynamique de ressources à la demande }
Contrairement aux grilles, les clouds donnent l'illusion que les ressources informatiques disponibles sont illimitées. Cela signifie que les utilisateurs peuvent demander, et s’attendre à obtenir des ressources suffisantes pour leurs besoins, à tout moment. L’approvisionnement à la demande est idéal pour les workflows et d'autres applications faiblement couplées, car il réduit le surcoût (overheads) d’ordonnancement total et peut améliorer considérablement les performances du workflow (Singh, 2005 ; Juve, 2008) 
\subsubsection{L’élasticité}
Outre l’approvisionnement des ressources à la demande, les clouds permettent aussi aux utilisateurs de libérer des ressources à la demande. La nature élastique de clouds facilite le changement des quantités et des caractéristiques de ressources lors de l'exécution, permettant ainsi d’augmenter le nombre de ressources, quand il y a un grand besoin, et d’en diminuer, lorsque la demande est faible. Cela permet aux systèmes de gestion de workflow de répondre facilement aux exigences de qualité de service (QoS) des applications, contrairement à l'approche traditionnelle, qui nécessite de réserver à l'avance des ressources dans les environnements de grilles. 
\subsubsection{La garantie des QoS via des SLA }
Avec l’arrivée des services de cloud computing  provenant de grandes organisations commerciales, les accords de niveau de service (SLA) ont été une préoccupation importante pour les fournisseurs et les utilisateurs. En raison de compétitions entres les fournisseurs de services émergents, un grand soin est pris lors de la conception du SLA qui vise à offrir (i) de meilleures garanties de QoS aux utilisateurs, et (ii) des termes clairs pour l'indemnisation, en cas de violation du contrat. Cela permet aux systèmes de gestion de workflow de fournir de meilleures garanties de bout en bout en "mappant" les utilisateurs aux fournisseurs de services selon les caractéristiques des SLA.

\subsubsection{Le faible Coût d’exploitation }
 Économiquement motivés, les fournisseurs de cloud commercial s'efforcent d'offrir de meilleures garanties de services par rapport aux fournisseurs de grille. Les fournisseurs de cloud profitent également des économies d'échelle, en fournissant des ressources de calcul, de stockage et de bande passante, à un coût très faible grâce, à la virtualisation. Ainsi l'utilisation des services de cloud public pourrait être économique et une alternative moins coûteuse, par rapport à l’utilisation de ressources dédiées, qui sont plus chères. Un des avantages de l'utilisation des ressources virtuelles pour l'exécution de workflow, plutôt que d'un accès direct à la machine physique, est le besoin réduit pour sécuriser les ressources physiques des codes malveillants. Cependant, l'effet à long terme de l'utilisation de ressources virtuelles dans les clouds qui partagent efficacement une "tranche" de la machine physique, plutôt que d'utiliser des ressources dédiées pour les workflows de calculs intensifs, est une question de recherche intéressante. 
\section{Conclusion}
....lk


%%%%
%\chapter{Réseaux de Petri}
%
  
 	 \section{Introduction}
 	  
    
    Le cloud computing, traduit le plus souvent en français par " informatique dans les nuages", " informatique dématérialisée " ou encore " infonuagique ", est un domaine qui regroupe un ensemble de techniques et de pratiques consistant à accéder, en libre-service, à du matériel ou à des logiciels informatiques, à travers une infrastructure réseau (Internet). Ce concept rend possible la distribution des ressources informatiques sous forme de services pour lesquels l'utilisateur paie uniquement pour ce qu'il utilise. Ces services peuvent être utilisés pour exécuter des applications scientifiques et commerciales, souvent modélisées sous forme de workflows.
     
    Ce chapitre présente une introduction au cloud computing et au workflow, nécessaire pour la compréhension générale de ce rapport.
    
    Tout d’abord, nous présentons dans la section 1.2 une introduction au paradigme du cloud computing. Nous donnons un aperçu général du cloud computing, y compris sa définition, ses caractéristiques principales et une comparaison avec les technologies connexes. Nous présentons les différents modèles de service, les différents modèles de déploiement, ainsi que les différents acteurs du cloud computing. Nous résumons quelques challenges de recherche en cloud computing. Par la suite, nous présentons, dans la section 1.3, une introduction au workflow et systèmes de gestion de workflow. Nous donnons le concept du workflow, sa définition, et l’architecture de référence d’un système de gestion de workflows. Nous énumérons quelques systèmes de gestion de workflows existant dans les grilles et clouds et, finalement, nous résumons l’intérêt  du cloud pour les workflows.
    
    \section{cloud computing}
    \subsection{Concept du cloud computing}
  L’idée principale du cloud est apparue dans les années 60, où le professeur John McCarthy avait imaginé que les ressources informatiques seront fournies comme des services d’utilité publique (Garfinkel, 1999). C'est ensuite, vers la fin des années 90, que ce concept a pris de l'importance avec l’avènement du grid computing  (Foster, 1999). Le terme cloud est une métaphore exprimant la similarité avec le réseau électrique, dans lequel l'électricité est produite dans de grandes centrales, puis disséminée à travers un réseau jusqu'aux utilisateurs finaux. Ici, les grandes centrales sont les Datacenter, le réseau est le plus souvent celui d'Internet et l'électricité correspond aux ressources informatiques. Le cloud computing  n'est véritablement apparu qu'au cours de l’année 2006 (Vouk, 2008) avec l'apparition d'Amazon EC2 (Elastic Compute cloud). C'est en 2009 que la réelle explosion du cloud survint avec l'arrivée sur le marché de sociétés comme Google (Google App Engine), Microsoft (Microsoft Azure), IBM (IBM Smart Business Service), Sun (Sun cloud) et Canonical Ltd (Ubuntu Enterprise cloud). D'après une étude menée par Forrester (Ried, 2011), le marché du cloud computing  s'élevait à environ 5,5 milliards de dollars en 2008, il devrait atteindre plus de 150 milliards d'ici 2020, comme l’illustre la figure \ref{fig:tempsnip4}. 
    
    \begin{figure}[h]
    	\centering
    	\includegraphics[width=0.7\linewidth]{images/tempsnip4}
    	\caption{Prévisions de la taille du marché du cloud computing  public (Ried, 2011).}
    	\label{fig:tempsnip4}
    \end{figure}
\subsubsection{Vers une définition du cloud computing }
Beaucoup de chercheurs ont tenté de définir le cloud computing (Geelan, 2008 ; McFedries, 2008 ; Buyya, 2009 ; Armbrust, 2010). La plupart des définitions attribuées à ce concept semblent se concentrer seulement sur certains aspects technologiques. L'absence d'une définition standard a généré non seulement des exagérations du marché, mais aussi des confusions. Pour cette raison, il y a eu récemment des travaux sur la normalisation de la définition du cloud computing, à l'exemple de Vaquero et coll (Vaquero, 2009) qui ont comparé plus de 20 définitions différentes et ont proposé une définition globale.  En guise de synthèse des différentes propositions données dans la littérature, nous introduisons une définition mixte, qui correspond aux différents types de cloud considérés dans les travaux réalisés dans cette thèse.

  Nous définissons le cloud comme un modèle informatique qui permet d’accéder, d’une façon transparente et à la demande, à un pool de ressources hétérogènes physiques ou virtualisées (serveurs, stockage, applications et services) à travers le réseau. Ces ressources sont délivrées sous forme de services reconfigurables et élastiques, à base d’un modèle de paiement à l’usage, dont les garanties sont offertes par le fournisseur via des contrats de niveau de service (SLA, Service Level Agreement).     
    
    \subsubsection{Caractéristiques principales du cloud computing}
    Le cloud computing  possède les caractéristiques suivantes :
    \begin{itemize}
    	\item 	\textbf{Accès en libre-service à la demande}. Le cloud computing offre des ressources et services aux utilisateurs à la demande. Les services sont fournis de façon automatique, sans nécessiter d’interaction humaine (Mell, 2011). 
    \item	\textbf{Accès réseau universel.}  Les services de cloud computing  sont facilement accessibles au travers du réseau, par le biais de mécanismes standard, qui permettent une utilisation depuis de multiples types de terminaux (par exemple, les ordinateur portables, tablettes, smartphones) (Mell, 2011). 
   \item \textbf{Mutualisation de ressources} (Pooling). Les ressources du cloud peuvent être regroupées pour servir des utilisateurs multiples, pour lesquels des ressources physiques et virtuelles sont automatiquement attribuées (Mell, 2011). En général, les utilisateurs n’ont aucun contrôle ou connaissance sur l’emplacement exact des ressources fournies. Toutefois, ils peuvent imposer de spécifier l’emplacement à un niveau d’abstraction plus haut.
   \item \textbf{Scalabilité et élasticité.} Des ressources supplémentaires peuvent être automatiquement mises à disposition des utilisateurs en cas d’accroissement de la demande (en réponse à l'augmentation des charges des applications) (Geelan, 2008), et peuvent être libérées lorsqu’elles ne sont plus nécessaires. L’utilisateur a l’illusion d’avoir accès à des ressources illimitées à n'importe quel moment, bien que le fournisseur en définisse généralement un seuil (par exemple : 20 instances par zone est le maximum possible pour Amazon EC2).
   \item \textbf{Autonome.} Le cloud computing  est un système autonome et géré de façon transparente pour les utilisateurs. Le matériel, le logiciel et les données au sein du cloud peuvent être 
    	automatiquement reconfigurés, orchestrés et consolidés en une seule image qui sera fournie à l’utilisateur (Wang, 2008).
    	\item \textbf{Paiement à l’usage.} La consommation des ressources dans le cloud s’adapte au plus près aux besoins de l’utilisateur. Le fournisseur est capable de mesurer de façon précise la consommation (en durée et en quantité) des différents services (CPU, stockage, bande passante,…) ; cela lui permettra de facturer l’utilisateur selon sa réelle consommation (Armbrust, 2009). 
    	\item \textbf{Fiabilité et tolérance aux pannes.} Les environnements cloud tirent parti de la redondance intégrée du grand nombre de serveurs qui les composent en permettant des niveaux élevés de disponibilité et de fiabilité pour les applications qui peuvent en bénéficier (Buyya, 2008). 
    	\item \textbf{Garantie QoS.} Les environnements de cloud peuvent garantir la qualité de service pour les utilisateurs, par exemple, la performance du matériel, comme la bande passante du processeur et la taille de la mémoire (Wang, 2008). 
    	\item \textbf{Basé-SLA.} Les clouds sont gérés dynamiquement en fonction des contrats d’accord de niveau de service (SLA) (Buyya, 2008) entre le fournisseur et l’utilisateur. Le SLA définit des politiques, telles que les paramètres de livraison, les niveaux de disponibilité, la maintenabilité, la performance, l'exploitation, ou autres attributs du service, comme la facturation, et même des sanctions en cas de violation du contrat. Le SLA permet de rassurer les utilisateurs dans leur idée de déplacer leurs activités vers le cloud, en fournissant des garanties de QoS. 
    	Après avoir présenté les caractéristiques essentielles d’un service cloud, nous présentons, brièvement, dans la section suivante, quelques technologies connexes aux clouds.
    	
    \end{itemize} 

%%%%%%%%%%%%%%%%%%%%%   Partie pratique  %%%%%%%%%%%%%%%%%%%%%%%%
%\part{  Partie pratique }


\chapter{Analyse et Conception}


\section*{Introduction}
Après avoir achevé notre étude de l’existant et étudié de près le système actuel, nous abordons l’étape suivante qui consiste à concevoir le nouveau système en utilisant la modélisation UML. L’objectif de cette étape est de déterminer de façon détaillée et précise ce que le nouveau système devrait faire, afin de répondre aux objectifs attendus. Dans ce chapitre nous allons détailler les objectifs fixés du système,en suite nous détaillons ses fonctionnalités et sa logique de fonctionnement. 
%Nous présentons aussi 

 \section{Unified Modeling Language (UML):}
 Dans ce qui suit, nous allons donner une brève description d’UML.
 \subsection{ Définition UML }
 L’OMG définit l’UML comme un langage visuel dédié à la spécification,la construction et la documentation des artéfacts d’un système logiciel. Aussi la façon dont tout le monde modélise non seulement la structure de l’application, le comportement et l’architecture, mais aussi des processus d’affaires et la structure des données. Ce langage est conçu pour modéliser divers types de systèmes et de taille quelconque. Il possède une approche entièrement objet couvrant tout le cycle de développement. Le système est décomposé en un ensemble d’objets collaborant [GUIBERT, 2010].
 
 \subsection{Contenu UML}
 
 L’UML comporte 13 diagrammes qui se répartissent en deux catégories [GUIBERT, 2010]. Nous ne mentionnons que les diagrammes que nous utilisons, les autres sont en annexe G. 
 
 \subsubsection{Diagramme structurel : }
 \begin{itemize}
 \item \textbf{Diagramme de classes (Class Diagram) : }ce diagramme décrit la structure statique du système, il définit les classes, leurs attributs et leurs relations. Il est considéré comme le diagramme le plus important.
  \item  \textbf{Diagramme de paquetages (Package Diagram) : }définit les dépendances entres les paquets (groupement d’éléments UML) constituant un modèle. 
 \end{itemize}
 
 
  \subsubsection{Diagramme comportemental  : }
 
  \begin{itemize}
 	\item \textbf{Diagramme de cas d’utilisation (Use Case Diagram) : } pour décrire les besoins des utilisateurs. 
 	\item  \textbf{Diagramme de séquence(Sequence Diagram):}  décrit comment chaque objet interagit avec l’autre et dans quel ordre, sur un axe temporel donné. Ce diagramme sont associés aux diagrammes de cas d’utilisation.
 	. 
 \end{itemize}



\section{Identification des objectifs}

Au début du projet, nous nous sommes concentrés sur les besoins qui pourraient normalement être considérés comme un système général et nous avons interrogé les ingénieurs de la société nationaux de retraités    "CNR", qui ont exprimé des besoins importants .
 
 
 Nos entretiens ont été complétés en suivant les étapes ci-dessous
 
\textbf{ Étape 1:} Choisir les entretiens Afin d'identifier les besoins, nous avons contacté des personnes pouvant fournir des informations utiles et fournir des explications. Les personnes responsables de la production des différentes unités ou entreprises sont les plus appropriées pour répondre à nos questions.
 
\textbf{Étape 2:}
Planification du développement du programme À ce stade de la planification des entretiens, nous avons étudié deux points essentiels:

- Déterminer le contexte général de l'entretien : "CNR",

- Réglage de la date et de l'heure de l'entretien (date) .

\textbf{ Étape 3:} Préparez-vous pour l’entrevue Préparez les questions de développement et les supports pour un processus sans faille (stylos, cartons de réponses, papiers blancs supplémentaires, clé USB).


\textbf{ Étape 4:} Conduisez l’entretien: Commencez le processus d’entretien en vous présentant d’abord, puis en présentant brièvement notre projet, puis en passant l’entretien en interrogeant la personne concernée et en enrichissant sa conversation d’observations et de questions. Intermédiaire.
 
\textbf{ Étape 5:} Après l'entretien Après avoir pris les informations collectées et conclu l'entretien, nous avons décidé de procéder à d'autres entretiens en développant et en nous réunissant à chaque fois.
 
 
 Sur la base de ces entretiens et de l'entretien avec l'enseignant, nous avons pu identifier les spécifications suivantes:
 
 \subsection{Spécifications fonctionnelles }
 
 \begin{enumerate}
\item  	   Le système doit permettre à l’administrateur  de  Gérer Modale .
\item  	   Le système doit permettre à l’administrateur  de  Gérer Sous-direction .
\item  	   Le système doit permettre à l’administrateur  de Gérer les service.
\item  	   Le système doit permettre à l’administrateur  de Gérer workflow pour les dossier  .
\item  	   Le système doit permettre de  Gérer les tâches des service . 
\item  	   Le système doit permettre à l’administrateur  de Gérer les utilisateur .
\item  	   Le système doit permettre à l’administrateur  de  consulte les historique  et la recherche par tâche .
\item  	   Le système doit permettre à l’administrateur  de  consulte les historique  et la recherche par dossier et les jours pour chaque tâche .
\item  	   Le système doit permettre à l’administrateur  de  consulte bordereau .
\item  	   Le système doit permettre à l’utilisateur   de  Gérer les dossier .
\item  	   Le système doit permettre à l’utilisateur   de  Activé le dossier    .
\item  	   Le système doit permettre à l’utilisateur   de  finir le traitement des dossiers par tâche    .
\item  	   Le système doit permettre à l’utilisateur   de  Gérer les  bordereaux  .
\item  	   Le système doit permettre à l’utilisateur   de  accepte le  bordereau  .
\item  	   Le système doit permettre à l’utilisateur   de  refuse  le  bordereau  .
\item  	   Le système doit permettre à l’utilisateur   de  imprimer le  bordereau  .

 \end{enumerate}
 
 
 
 
  \subsection{Spécifications techniques }
 
 
 
  \begin{enumerate}
 	\item  	   Le système doit être déployé sur le cloud  .
 	\item  	    la configuration et la germanisation de code dynamique . 
 	\item  	  la modélisation de workflow par dessiner  BPMN
 	
 
 	
 \end{enumerate}
 
 
 
 
 
 
 
 
 
 
 
 \subsection{ Cas d’utilisation }
 Le diagramme des cas d’utilisation suivant illustre les fonctionnalités qu’un simple utilisateur du système, également l’administrateur, peut faire. Ce diagramme a été inspiré des spécifications citées ci-dessus :
 
 \begin{figure}[H]
 	\centering
 	\includegraphics[width=1\linewidth]{images/usercase}
 	\caption{Diagramme de Cas d’utilisation}
 	\label{fig:usercase}
 \end{figure}
 
 \subsection{ Documentation des cas d’utilisation fonctionnels }
 \subsubsection{ S’authentifier }
 \begin{table}[H]
 	\centering
 	\begin{tabular}{|l|} 
 		\hline
 	\textbf{	CU :} S’authentifier     \\  	\hline
 		\textbf{ID:}1         \\  		\hline
 		\begin{tabular}[c]{@{}l@{}}\textbf{Description brève :} chaque utilisateur doit s’authentifier\\ auprès de l’application afin de pouvoir \\ utiliser les fonctionnalités du système, tel que la consultation du\\ tableau de bord, la manipulation les rapports ...etc. \end{tabular}          \\ 
 		\hline
 		\textbf{Acteurs primaires : }utilisateur, administrateur   \\ 
 		\hline
 		Acteurs secondaires :                                                                                                                                                                                                                                                                                                                                                                                                                                                                                                                                                                                                                                                                                                                                                                             \\ 
 		\hline
 		\begin{tabular}[c]{@{}l@{}}\textbf{Pré-conditions :} – La connexion auprès du serveur d’application doit\\ être réussite. – L’utilisateur \\ doit être enregistré dans le système. \end{tabular}                                                                                                                                                                                                                                                                                                                                                                                                                                                                                                                                                                                                           \\ 
 		\hline
 		\begin{tabular}[c]{@{}l@{}}\textbf{Enchainement principal : }Ce cas d’utilisation commence lorsqu’un\\ utilisateur souhaite accéder à \\ l’application. \\ 1. L’utilisateur saisit le lien de l’application dans barre\\ d’adresse du navigateur. \\ 2. Le serveur répond à l’utilisateur en renvoyant une page\\ d’authentification.\\ 3. L’utilisateur saisit\\ son nom d’utilisateur et son mot de passe et les valide en appuyant sur \\ 4. “Log in”. \\ 5. Le serveur vérifie la validité du nom d’utilisateur et du mot\\ de passe. \\ 6. Le serveur envoie une page d’accueil de l’application à\\ l’utilisateur concerné. \end{tabular}     \\ 
 		\hline
 		Post-conditions : L’utilisateur est connecté.                                                                                     \\ 
 		\hline
 		\begin{tabular}[c]{@{}l@{}}Enchainement alternatif :  \\\begin{tabular}{@{\labelitemi\hspace{\dimexpr\labelsep+0.5\tabcolsep}}l} E1 : La page d’authentification n’apparait pas à l’utilisateur.\end{tabular}\\ * L’enchainement démarre   après le premier point de l’enchainement principal. \\ * Le serveur envoie un message d’erreur à  l’utilisateur. \\\begin{tabular}{@{\labelitemi\hspace{\dimexpr\labelsep+0.5\tabcolsep}}l} E2 : Le nom d’utilisateur ou/et le mot de passe ne sont pas valides.\end{tabular}\\ L’enchainement démarre après le quatrième point de l’enchainement principal. 
 			\\
 			* Le serveur envoie un message d’erreur à l’utilisateur.
 			
 			\\ * Le serveur demande à l’utilisateur de ressaisit le nom d’utilisateur et le mot de passe. \end{tabular}  \\
 		\hline
 	\end{tabular}
 \end{table}
 
  \subsubsection{Créer Dossiers}
 \begin{table}[H]
 	\begin{tabular}{|l|}
 		\hline
 	\textbf{	CU : }Créer   Dossiers \\ \hline
 	\textbf{	ID }: 2 \\ \hline
 	\textbf{	Description brève :} chaque utilisateur pus Créer  et enregistre les noves   Dossiers \\ \hline
 	\textbf{	Acteurs primaires :} utilisateur réception ou initiale, \\ \hline
 		\textbf{Acteurs secondaires :} simple utilisateur \\ \hline
 	\textbf{	Pré-conditions :} L’utilisateur doit être  connecte ou  le système. \\ \hline
 		\begin{tabular}[c]{@{}l@{}}\textbf{Enchainement principal :} Ce cas d’utilisation commence   lorsqu’un utilisateur\\  souhaite accéder à  l’application. \\   1.      L’utilisateur sélectionner de ajouté Nouvo dossier. \\   2.      system répond à l’utilisateur en afficher un formuler de enregistrement.\\   3.       L’utilisateur saisit le   code  et le nom prénom et numéro de typhon\\   de personne de dossier    \\   4.      L’utilisateur vérifier le dossier  et  sélectionner   les chants  présent.\\  en appuyant sur « enregistrer ».\\   5.      Le system envoie une page des dossiers.\end{tabular} \\ \hline
 		\textbf{Post-conditions : }le dossier   est enregistré et ajouté dans le system. \\ \hline
 		\begin{tabular}[c]{@{}l@{}}\textbf{Enchainement alternatif : }    \\  \textbf{ E1:} Le code  de Dossier n’est pas valides ou/et déjà existe.\\   o    L’enchainement démarre après le quatrième point de l’enchainement principal. \\   o     Le serveur envoie un message d’erreur à  l’utilisateur. \\   o   Le serveur demande à l’utilisateur de ressaisit le code de Dossier.\end{tabular} \\ \hline
 	\end{tabular}
 \end{table}
 
 




\subsubsection{Activer Dossiers}
\begin{table}[H]
	\begin{tabular}{|l|}
		\hline
		\textbf{	CU : }Activer Dossiers \\ \hline
		\textbf{	ID }: 3 \\ \hline
		\textbf{	Description brève :}L'utilisateur peut activer le statut du dossier et commencer \\au  début de la tâche "Ajouter au flux de travail" ,\\ après avoir rempli tous les documents.    \\ \hline
		\textbf{	Acteurs primaires :} utilisateur réception ou initiale, \\ \hline
		\textbf{Acteurs secondaires :} simple utilisateur \\ \hline
		\textbf{	Pré-conditions :} Le Dossier  doit être enregistré dans le système.\\ \hline
		\begin{tabular}[c]{@{}l@{}}\textbf{Enchainement principal :} \\ 1.      L’utilisateur sélectionner de ajouté la liste des dossier . \\ 
			2.      system répond à l’utilisateur en afficher  les liste des dossier \\ \{"tous la liste","liste activer ","liste non activer" \} .\\  
			3.       L’utilisateur sélectionner  dossier    \\ 
			4. le system  répond à l’utilisateur en afficher le détaille, est aprés\\  en appuyant sur "Activer". \\  
			5. le system vérifie id "code" de dossier et activer leur état \end{tabular} \\ \hline
		\textbf{Post-conditions : }le dossier  est activé dont la premier tache . \\ \hline
		\begin{tabular}[c]{@{}l@{}}\textbf{Enchainement alternatif : } \end{tabular} \\ \hline
	\end{tabular}
\end{table}





 
 
\subsubsection{Consulter état des Dossiers}
\begin{table}[H]
	\begin{tabular}{|l|}
		\hline
		\textbf{	CU : }Consulter état des Dossiers \\ \hline
		\textbf{	ID }: 4 \\ \hline
		\textbf{	Description brève :}L'utilisateur peut visualiser le statut des dossiers en cours \\ de traitement  en fonction de sa tâche et voir tous les dossiers envoyés par une \\autre tache.     \\ \hline
		\textbf{	Acteurs primaires :} utilisateur  \\ \hline
		\textbf{Acteurs secondaires :}  \\ \hline
		\textbf{	Pré-conditions :} 
		\\ - Le Dossier  doit être Activé dans le système.
		\\ -  L’utilisateur doit être  connecte ou  le système.
		\\ \hline
		\begin{tabular}[c]{@{}l@{}}\textbf{Enchainement principal :} \\ 1.      L’utilisateur sélectionner  le statut des dossiers . \\ 
			2.      system répond à l’utilisateur en afficher  les liste des dossier encore de traitement\\ ou les noves  dossiers envoyés    \end{tabular} \\ \hline
		\textbf{Post-conditions : }L'utilisateur   visualiser le statut  des dossiers par sa tâche. \\ \hline
		\begin{tabular}[c]{@{}l@{}}\textbf{Enchainement alternatif : } \end{tabular} \\ \hline
	\end{tabular}
\end{table}



\subsubsection{Finir le traitement  des Dossiers}
\begin{table}[H]
	\begin{tabular}{|l|}
		\hline
		\textbf{	CU : }Finir le traitement  des Dossier \\ \hline
		\textbf{	ID }: 5 \\ \hline
		\textbf{	Description brève :}L'utilisateur peut sélectionner liste des dossier et finir\\ leur traitement   \\ \hline
		\textbf{ Acteurs primaires :} utilisateur  \\ \hline
		\textbf{Acteurs secondaires :}  \\ \hline
		\textbf{	Pré-conditions :} 
		\\ - Le Dossier  doit être Activé dans    la tâche de l'utilisateur .
		\\ -  L’utilisateur doit être  connecte ou  le système.   
		\\ \hline
		\begin{tabular}[c]{@{}l@{}}\textbf{Enchainement principal :} \\ 1.      L’utilisateur sélectionner la liste  des dossiers pour finir le traitement en payent \\sur "terminer"  , \\ 
			2.      system répond à l’utilisateur et terminer le traitement des dossier par \\ changer état et la date fin de traitement .  \end{tabular} \\ \hline
		\textbf{Post-conditions : }L'utilisateur   finir le traitement   des dossiers par sa tâche. \\ \hline
		\begin{tabular}[c]{@{}l@{}}\textbf{Enchainement alternatif : } \end{tabular} \\ \hline
	\end{tabular}
\end{table}






\subsubsection{Gérer Bordereau}
\begin{table}[H]
	\begin{tabular}{|l|}
		\hline
		\textbf{	CU : }Gérer Bordereau   \\ \hline
		\textbf{	ID }: 6 \\ \hline
		\textbf{	Description brève :}L'utilisateur peut sélectionner liste des dossier et \\Gérer un  Bordereau  \\   \hline
		\textbf{ Acteurs primaires :} utilisateur  \\ \hline
		\textbf{Acteurs secondaires :}  \\ \hline
		\textbf{	Pré-conditions :} 
		\\ - Le Dossier  doit être terminer le traitement  dans    la tâche de l'utilisateur  .
		\\ -  L’utilisateur doit être  connecte ou  le système.   
		\\ \hline
		\begin{tabular}[c]{@{}l@{}}\textbf{Enchainement principal :} \\ 1.      L’utilisateur sélectionner la liste  des dossiers et sélectionner \\le suivant tache  pour transfert    \\ 
			2.  system répond à l’utilisateur et créer  nounous bordereau et afficher la liste\\ de bordereau correspondre a leur tache .
		  \\ 
		3.  L’utilisateur put imprimer et accepte ou refuse le bordereau  .   \end{tabular} \\ \hline
		\textbf{Post-conditions : } L'utilisateur  créer bordereau et transfert les dossier \\si le bordereau accepte par validation d'un utilisateur de la tâche reçu . \\ \hline
		\begin{tabular}[c]{@{}l@{}}\textbf{Enchainement alternatif : }  
	\textbf{	E1:bordereau refuse. }
	\\ * renvoyer a la tâche président tous les dossier   
	
 \end{tabular} \\ \hline
	\end{tabular}
\end{table}
 
 \section{ Représentation des informations }
 \subsection{Diagramme de Classe}
\begin{figure}[H]
	\centering
	\includegraphics[width=1\linewidth]{images/class01}
	\caption{Diagramme de Classe}
	\label{fig:class01}
\end{figure}

  \subsubsection{Conception détaillée} 
Nous allons définir pour chaque classe ses attributs et leur types , ainsi que les méthodes qu’elle offre. Dans le but d’alléger le rapport, nous avons jugé essentiel de ne citer  les classes que nous avons conçu  .
 
 

\subsubsection*{La classe  Sous-Direction}
\begin{table}[H]
  \centering\setlength\tabcolsep{0.8cm}
	\begin{tabular}{|l|l|l|}
		\hline
		\textbf{Attribut}  & \textbf{Type} & \multicolumn{1}{l|}{\textbf{Méthodes}} \\ \hline
	
		id & long & getId() et setId()\\ \cline{1-2}
		nom & String  & getNom() et setNom() \\ \cline{1-2}
	services	& $ List<Service> $ & getServices() et setServices()   \\ \hline
	\end{tabular}
\end{table}
\subsubsection*{La classe Service}
\begin{table}[H]
	\centering
	\begin{tabular}{|l|l|l|}
		\hline
		\textbf{Attribut}  & \textbf{Type} & \multicolumn{1}{l|}{\textbf{Méthodes}} \\ \hline
		
		id & long & getId() et setId()\\ \cline{1-2}
		nom & String  & getNom() et setNom() \\ \cline{1-2}
				sous-Direction & Sous-Direction  & getSous-Direction() et setSous-Direction() \\ \cline{1-2}
		myTasks	& $ List<MyTask> $ & getMyTasks() et setMyTasks()   \\ \hline
	\end{tabular}
\end{table}


\subsubsection*{La classe MyTask}
\begin{table}[H]
	\centering
	\begin{tabular}{|l|l|l|}
		\hline
		\textbf{Attribut}  & \textbf{Type} & \multicolumn{1}{l|}{\textbf{Méthodes}} \\ \hline
		
		id & long & getId() et setId()\\ \cline{1-2}
		nom & String  & getNom() et setNom() \\ \cline{1-2}
			idTask & String & getIdTask() et setIdTask()\\ \cline{1-2}
		services & Service  & getService() et setService() \\ \cline{1-2}
				users	& $ List<User> $ & getUsers() et setUsers()   \\ \cline{1-2}	
				
		nextMyTasks	& $ List<NextMyTask> $ & getNextMyTasks() et setNextMyTasks()   \\ \hline
	\end{tabular}
\end{table}

\subsubsection*{La classe MyTask}
\begin{table}[H]
	\centering
	\begin{tabular}{|l|l|l|}
		\hline
		\textbf{Attribut}  & \textbf{Type} & \multicolumn{1}{l|}{\textbf{Méthodes}} \\ \hline
		
		id & long & getId() et setId()\\ \cline{1-2}
		nom & String  & getNom() et setNom() \\ \cline{1-2}
		idTask & String & getIdTask() et setIdTask()\\ \cline{1-2}
		services & Service  & getService() et setService() \\ \cline{1-2}
		nextMyTasks	& $ List<NextMyTask> $ & getNextMyTask() et setNextMyTask()   \\ \hline
	\end{tabular}
\end{table}

\subsubsection*{La classe NextTask}
\begin{table}[H]
  \centering\setlength\tabcolsep{1cm}

	\begin{tabular}{|l|l|l|}
		\hline
		\textbf{Attribut}  & \textbf{Type} & \multicolumn{1}{l|}{\textbf{Méthodes}} \\ \hline
		
		id & long & getId() et setId()\\ \cline{1-2}
		idTask & String & getIdTask() et setIdTask()\\ \cline{1-2}
		myTask & MyTask  & getMyTask() et setMyTask()   \\ \hline
	\end{tabular}
\end{table}

\subsubsection*{La classe Utilisateur "User"}
\begin{table}[H]
	\centering\setlength\tabcolsep{1cm}
	
	\begin{tabular}{|l|l|l|}
		\hline
		\textbf{Attribut}  & \textbf{Type} & \multicolumn{1}{l|}{\textbf{Méthodes}} \\ \hline
		
		id & long & getId() et setId()\\ \cline{1-2}
		nom & String & getNom() et setNom()\\ \cline{1-2}
			prenom & String & getPrenom() et setPrenom()\\ \cline{1-2}
						tlphon & String & getTlphon() et setTlphon()\\ \cline{1-2}
		myTask & MyTask  & getMyTask() et setMyTask()   \\ \hline
	\end{tabular}
\end{table}




\subsubsection*{La classe Role}
\begin{table}[H]
	\centering\setlength\tabcolsep{1.2cm}
	
	\begin{tabular}{|l|l|l|}
		\hline
		\textbf{Attribut}  & \textbf{Type} & \multicolumn{1}{l|}{\textbf{Méthodes}} \\ \hline
		
		id & long & getId() et setId()\\ \cline{1-2}
		nom & Date() & getNom() et setNom()\\ \hline
	\end{tabular}
\end{table}






\subsubsection*{La classe Dossier}
\begin{table}[H]
	\centering\setlength\tabcolsep{1cm}
	
	\begin{tabular}{|l|l|l|}
		\hline
		\textbf{Attribut}  & \textbf{Type} & \multicolumn{1}{l|}{\textbf{Méthodes}} \\ \hline
		
		id & long & getId() et setId()\\ \cline{1-2}
		nom & String & getNom() et setNom()\\ \cline{1-2}
		prenom & String & getPrenom() et setPrenom()\\ \cline{1-2}
		tlphon & String & getTlphon() et setTlphon()\\ \cline{1-2}
		ch1 & boolean  & getCh1() et setCh1()\\ \cline{1-2}   
				ch2 & boolean  & getCh2() et setCh2() \\ \cline{1-2}
						ch3 & boolean  & getCh3() et setCh3() 
		\\ \hline
	\end{tabular}
\end{table}



\subsubsection*{La classe ListDossier}
\begin{table}[H]
	\centering\setlength\tabcolsep{0.8cm}
	
	\begin{tabular}{|l|l|l|}
		\hline
		\textbf{Attribut}  & \textbf{Type} & \multicolumn{1}{l|}{\textbf{Méthodes}} \\ \hline
		
		id & long & getId() et setId()\\ \cline{1-2}
		bordereau & Bordereau & getBordereau() et setBordereau()\\ \cline{1-2}
		historique & Historique & getHistorique() et setHistorique()\\   \hline
	\end{tabular}
\end{table}


\subsubsection*{La classe Historique}
\begin{table}[H]
	\centering\setlength\tabcolsep{1cm}
	
	\begin{tabular}{|l|l|l|}
		\hline
		\textbf{Attribut}  & \textbf{Type} & \multicolumn{1}{l|}{\textbf{Méthodes}} \\ \hline
		
		id & long & getId() et setId()\\ \cline{1-2}
		dateD & Date() & getDateD() et setDateD()\\ \cline{1-2}
		
				dateF & Date() & getDateF() et setDateF()\\ \cline{1-2}
				
				dossier & Dossier & getDossier() et setDossier()\\ \cline{1-2}
				
					user & User & getUser() et setUser()\\ \cline{1-2}
					
					etat & boolean & isEtat() et setEtat()\\ \cline{1-2}
				transfert & boolean & isTransfert() et setTransfert()\\ \hline
	\end{tabular}
\end{table}
 
   
   \subsubsection*{La classe Bordereau}
   \begin{table}[H]
   	\centering\setlength\tabcolsep{1cm}
   	
   	\begin{tabular}{|l|l|l|}
   		\hline
   		\textbf{Attribut}  & \textbf{Type} & \multicolumn{1}{l|}{\textbf{Méthodes}} \\ \hline
   		
   		id & long & getId() et setId()\\ \cline{1-2}
   		dateD & Date() & getDateD() et setDateD()\\ \cline{1-2}
   	    
   	    user & User & getUser() et setUser()\\ \cline{1-2}
   				myTask1 & MyTask  & getMyTask1() et setMyTask1()   \\ \cline{1-2}
   						myTask2 & MyTask  & getMyTask2() et setMyTask2() \\ \cline{1-2}  
   		etat & String[] & getEtat() et setEtat()\\ \cline{1-2}
		historique & Historique & getHistorique() et setHistorique()\\ \hline
   	\end{tabular}
   \end{table}
 
   


 \chapter{Réalisation }
 \section{Outils}
 \subsection{Spring}
 
  \subsubsection{Spring Framework }
  
  
  Spring Framework fournit un modèle complet de programmation et de configuration pour les applications d'entreprise modernes basées sur Java - sur tout type de plate-forme de déploiement.
  
  Un élément clé de Spring est le support infrastructurel au niveau de l’application: Spring met l’accent sur la «plomberie» des applications d’entreprise afin que les équipes puissent se concentrer sur la logique métier au niveau de l’application, sans liens inutiles avec des environnements de déploiement spécifiques.
  
 \subsubsection{Spring boot}
 Spring Boot makes it easy to create stand-alone, production-grade Spring based Applications that you can "just run".
 
 We take an opinionated view of the Spring platform and third-party libraries so you can get started with minimum fuss. Most Spring Boot applications need very little Spring configuration.
   \subsubsection{Spring Data}
   
 \subsubsection{Spring Cloud}
 Spring Cloud fournit aux développeurs des outils permettant de créer rapidement certains modèles courants dans les systèmes distribués (gestion de la configuration, découverte de services, disjoncteurs, routage intelligent, micro-proxy, bus de contrôle, jetons à usage unique, verrous globaux, élection des dirigeants, distribution sessions, état du cluster). La coordination des systèmes distribués conduit à des modèles de plaque de chaudière et, grâce à Spring Cloud, les développeurs peuvent rapidement mettre en service des services et des applications mettant en œuvre ces modèles. Ils fonctionneront bien dans n’importe quel environnement distribué, y compris l’ordinateur portable du développeur, les centres de données à nu et les plateformes gérées telles que Cloud Foundry.
 
 
  \subsubsection{Spring Cloud Netflix}
 Spring Cloud Netflix fournit des intégrations Netflix OSS pour les applications Spring Boot via la configuration automatique et la liaison à Spring Environment et à d'autres idiomes de modèles de programmation Spring. Quelques annotations simples vous permettent d'activer et de configurer rapidement les modèles courants dans votre application et de créer de grands systèmes distribués avec des composants Netflix testés au combat. Les modèles fournis incluent la découverte de service (Eureka), le disjoncteur (Hystrix), le routage intelligent (Zuul) et l’équilibrage de la charge côté client (ruban).
 
 \subsection{ Qu'est-ce que Micro Service?}

 L'objectif principal de la mise en œuvre des micro-services est de scinder l'application en un service distinct pour chaque fonctionnalité de base et chaque service de l'API. Elle devrait être déployée indépendamment sur le cloud. Nous avons choisi le langage de programmation réactif du projet familial spring.io avec un ensemble de composants pouvant être utilisés pour mettre en œuvre notre modèle d'exploitation. Spring Cloud intègre très bien les composants Netflix dans l’environnement Spring. Il utilise une configuration automatique et une convention de configuration similaire à celle du fonctionnement de Spring Boot.


\subsubsection{Pourquoi l’architecture de Microservices?}

Nous avons choisi l'architecture de micro-services pour écrire chaque fonctionnalité en tant que service distinct pour les fonctionnalités de base et d'API, ce qui nous aide à réaliser la livraison et l'intégration en continu.

\subsubsection{Patterns dans l'architecture des microservices}



\begin{itemize}
\item  \textbf{Api Gatway }


\begin{enumerate}
	
\item 	  Choisissez de créer l’application en tant qu’ensemble de micro-services.

\item	  Décidez comment le client de l'application va interagir avec les micro-services.

\item	  Avec une application monolithique, il n'y a qu'un seul ensemble de points d'extrémité (généralement répliqués, à charge équilibrée).

\item	  Dans une architecture de micro-services, chaque micro-service expose cependant un ensemble de
points finaux.
\end{enumerate}


\item \textbf{Service registry}

\begin{enumerate}
\item 	  Le registre de service aide à déterminer l’emplacement des instances de service pour envoyer la demande au correspondant
	un service
	
\item 	  Ici, nous avons utilisé Netflix Eureka pour enregistrer un service pouvant être enregistré dans le registre de services.
	serveur et il peut être identifié par le routeur.

	
\end{enumerate}
\item \textbf{Service Discovry}
\begin{enumerate}


\item   Dans une application monolithique, les services s'appellent par le biais d'appels de méthode ou de procédure au niveau de la langue.

\item   Toutefois, dans une application moderne basée sur des micro-services, elle s'exécute généralement dans des environnements virtualisés où le nombre
des instances d'un service et de leurs emplacements change de façon dynamique.

\item   Chaque service peut être identifié à l'aide d'un routeur enregistré auprès du serveur de registre de services.
	
\end{enumerate}
\end{itemize}






\subsubsection{Architecture des microservices via les composants Netflix}
Nous avons utilisé les composants Netflix pour réaliser les modèles d’architecture de microservices ci-dessus.

% Please add the following required packages to your document preamble:
% \usepackage[table,xcdraw]{xcolor}
% If you use beamer only pass "xcolor=table" option, i.e. \documentclass[xcolor=table]{beamer}
% \usepackage[normalem]{ulem}
% \useunder{\uline}{\ul}{}
\begin{table}[H]
	\begin{tabular}{|l|l|}
		\hline
		\rowcolor[HTML]{5174DA} 
		{\color[HTML]{FFFFFF} \textbf{Operations Component}} & {\color[HTML]{FFFFFF} \textbf{Spring, Netflix OSS}} \\ \hline
		Service Discovery server & Netflix Eureka \\ \hline
		Edge Server & Netflix Zuul \\ \hline
		Central configuration server & Spring Cloud Config Server \\ \hline
		Dynamic Routing and Load Balancer & Netflix Ribbon \\ \hline
		OAuth 2.0 protected API’s & Spring Cloud + Spring Security OAuth2 \\ \hline
		Monitoring & Netflix Hystrix dashboard and turbine \\ \hline
	\end{tabular}
\end{table}

\subsection{Composantes majeures de Netflix}



\subsubsection{Service Discovery Server} 

\authorimg{images/ms-img04} Netflix Eureka permet aux micro-services de s’enregistrer eux-mêmes au moment de leur exécution, tels qu’ils apparaissent dans la structure du système.



\subsubsection{Routage dynamique et équilibreur de charge} 

\authorimg{images/ms-img05}  Netflix Ribbon peut être utilisé par les consommateurs de services pour rechercher des services au moment de l’exécution. Le ruban utilise les informations disponibles dans Eureka pour localiser les instances de service appropriées. Si plusieurs instances sont trouvées, le Ruban appliquera un équilibrage de charge pour répartir les demandes sur les instances disponibles. Le ruban ne s'exécute pas en tant que service distinct, mais en tant que composant intégré dans chaque consommateur de service.





\subsubsection{Serveur Edge} 

\authorimg{images/ms-img06}  Zuul est (bien sûr) notre gardien du monde extérieur, ne permettant pas le passage de demandes externes non autorisées. Zulu fournit également un point d'entrée bien connu aux micro-services dans le paysage système. L'utilisation de ports alloués de manière dynamique est pratique pour éviter les conflits de ports et minimiser l'administration, mais elle rend évidemment la tâche plus difficile pour tout consommateur de services donné. Zuul utilise Ribbon pour rechercher les services disponibles et achemine la demande externe vers une instance de service appropriée.


\subsection{Spring Boot et Spring Cloud Netflix OSS – Micro Service Architecture}

\subsubsection{Micro Services avec Spring Boot}

 

Spring Boot est un tout nouveau framework de l'équipe de Pivotal, conçu pour simplifier le démarrage et le développement d'une nouvelle application Spring. Le cadre adopte une approche de configuration avisée, libérant les développeurs de la nécessité de définir la configuration standard.

\subsubsection{Spring Cloud Netflix}
 

Spring cloud Netflix fournit des intégrations Netflix OSS pour les applications de démarrage printanier via la configuration automatique et la liaison à l'environnement Spring et à d'autres modèles de programmation Spring. Avec quelques annotations simples, nous pouvons rapidement activer et configurer des modèles courants dans une application et construire des systèmes distribués volumineux avec des composants Netflix. De nombreuses fonctionnalités sont disponibles avec le nuage de printemps Netflix. Ici, nous avons répertorié certaines des fonctionnalités communes que nous avons implémentées avec les micro-services avec Spring Boot et Netflix,

\begin{itemize}
\item \textbf{Découverte du service:}
	
	Les instances Eureka peuvent être enregistrées et les clients peuvent
	Découvrez les exemples à l'aide de haricots à gestion printanière
	

\item \textbf{	Création de service:}	
	Le serveur Eureka intégré peut être créé avec
	configuration Java déclarative
	
\item \textbf{	Configuration Externel:}	
	
	Bridge from the Spring Environment (permet aux utilisateurs de
	configuration des composants Netflix à l'aide de Spring Boot
	conventions)
	
      \item \textbf{	Routeur et filtre:}	
	
	Enregistrement automatique des filtres Zuul, et un simple
	convention sur l'approche de configuration pour la création de proxy inverse.
\end{itemize}
 
 
 
%\printbibliography %[title={Whole bibliography}]
   
   \chapter*{Conclusion générale }
\large
Dans ce dernier projet, nous sommes intéressés à développer le travail des entreprises et organisations modernes en Algérie, par le développement d'un système qui fournit des services permettant d’accélérer, l’organisation et contrôle les activités et le travail de ces entreprises et organisations.

L’objectif de ce projet est de conception et les mises en pales d'un service cloud qui fournit aux organisations et les entreprises un système workflow administrative . 

Afin d’atteindre cet objectif, nous avons commencé à définir l’approche en matière de direction de projet: il était tout d’abord nécessaire d’établir un état des connaissances dans le cloud et dans le workflow, puis nous avons procédé à une analyse des besoins, puis conçu et développé cette solution.



La meilleure solution consiste à répondre aux besoins spécifiques en fournissant des services flexibles sous la forme d'un flux de travail , permettant à l'automatisation des opérations de l'organisation de formaliser, normaliser et accélérer le traitement de documents et d'autres tâches.
\normalsize
  
    \comment{  \section{First section}
  
  This document is an example of \texttt{thebibliography} environment using 
  in bibliography management. Three items are cited: \textit{The \LaTeX\ Companion} 
  book \cite{latexcompanion}, the Einstein journal paper \parencite{einstein}, and the 
  Donald Knuth's website \cite{knuthwebsite}. The \LaTeX\ related items are
  \cite{latexcompanion,knuthwebsite}. 

  \medskip

  \begin{thebibliography}{9}
  	\bibitem{latexcompanion} 
  	Michel Goossens, Frank Mittelbach, and Alexander Samarin. 
  	\textit{The \LaTeX\ Companion}. 
  	Addison-Wesley, Reading, Massachusetts, 1993.
  	
  	\bibitem{einstein} 
  	Albert Einstein. 
  	\textit{Zur Elektrodynamik bewegter K{\"o}rper}. (German) 
  	[\textit{On the electrodynamics of moving bodies}]. 
  	Annalen der Physik, 322(10):891–921, 1905.
  	
  	\bibitem{knuthwebsite} 
  	Knuth: Computers and Typesetting,
  	\\\texttt{http://www-cs-faculty.stanford.edu/\~{}uno/abcde.html}
  	
  	
  	\bibitem{VAN DER AALST 97a}
  	 Van-der-Aalst, W.M.P. Verification of Workflow Nets. Application and Theory of Petri Nets, Lecture Notes in Computer Science Springer - Verlag 1248 (In P. Azema and G. Balbo, editors, ). 407 - 426, 1997
  \end{thebibliography}
}
 

 \printbibliography   
 \fancyhead{}

\appendix

 	\pagenumbering{roman} % switch to roman numerals
 \fancyhead[LO,CE]{Annexe: \thechapter}
\chapter{Service 1}
\section{classe User implémente en java  }

\begin{lstlisting}[language=java]
 import javax.persistence.Entity;
 import javax.persistence.GeneratedValue;
 import javax.persistence.Id;
 import java.io.Serializable;
 
 @Entity
 public class User implements Serializable {
 @Id
 @GeneratedValue
 private int id;
 private String name;
 public user(){
 }
 public user(String name) {
 this.name = name;
 }
 public int getId() {
 return id;
 }
 public void setId(int id) {
 this.id = id;
 }
 public String getName() {
 return name;
 }
 public void setName(String name) {
 this.name = name;
 }
 }
\end{lstlisting}

\section{créer l'interface UserRepository}


\begin{lstlisting}[language=java]
import org.springframework.data.jpa.repository.
JpaRepository;

public interface userRepository extends JpaRepository<user , Integer>{ 
}
\end{lstlisting}


\section{classe DummyDataCLR en java }
\begin{lstlisting}[language=java]
 

import org.springframework.beans.factory.annotation.Autowired;
import org.springframework.boot.CommandLineRunner;
import org.springframework.stereotype.Component;

import java.util.stream.Stream;

@Component
class DummyDataCLR implements CommandLineRunner {

@Override
public void run(String... strings) throws Exception {
Stream.of("Pencil", "Book", "Eraser").forEach(s->userRepository.save(new user(s)));
userRepository.findAll().forEach(s->System.out.println(s.getName()));
}

@Autowired
private userRepository userRepository;

}
\end{lstlisting}

\chapter{ConfigService}
 
\section{classe ConfigServiceApplication en java }
   \begin{lstlisting}[language=java]
   import org.springframework.boot.SpringApplication;
   import org.springframework.boot.autoconfigure.
   SpringBootApplication;
   import org.springframework.cloud.config.server.
   EnableConfigServer;
   
   @EnableConfigServer
   @SpringBootApplication
   public class ConfigServiceApplication {
   
   public static void main(String[] args) {
   
   SpringApplication.run(ConfigServiceApplicatin.class, args);
   }
   }
   \end{lstlisting} 
   
 \ 
  
 \section{json }
       \begin{lstlisting} 
       {
       name: "service-1",
       profiles: [
       "master"
       ],
       label: null,
       version: "6e1ea61d706133e2d8b62f40c6b784192fb58e8a",
       state: null,
       propertySources: [
       {
       name: "file:./src/main/resources/myConfig/application.properties",
       source: {
       global: "xxxxx"
       }
       }
       ]
       }
         \end{lstlisting}
        
          \section{json }
              \begin{lstlisting}
        {
        	name: "service-1",
        	profiles: [
        	"master"
        	],
        	label: null,
        	version: "6e1ea61d706133e2d8b62f40c6b784192fb58e8a",
        	state: null,
        	propertySources: [
        	{
        		name: "file:./src/main/resources/myConfig/user-service.properties",
        		source: {
        			me: "Djamel.Zerrouki@jimmi.fr"
        		}
        	},
        	{
        		name: "file:./src/main/resources/myConfig/application.properties",
        		source: {
        			global: "xxxxx"
        		}
        	}
        	]
        } 
        \end{lstlisting}
        
       \section{classe userRestService en java }
      \begin{lstlisting}
import  org.springframework.beans.factory.annotation.Value;
import org.springframework.web.bind.annotation.RequestMapping;
import org.springframework.web.bind.annotation.RestController;

@RestController
public class userRestService {

@Value("${me}")
private String me;

@RequestMapping("/messages")
public String tellMe(){
System.out.println("c'est moi qui ai repondu!");
return me;
}
}
      \end{lstlisting}  
        
        \chapter{Microservice DiscoveryService}
         
           \section*{EurekaServer }
        \begin{lstlisting}
        
        package tn.insat.tpmicro.discoveryservice;
        
        import org.springframework.boot.SpringApplication;
        import org.springframework.boot.autoconfigure.SpringBootApplication;
        import org.springframework.cloud.netflix.eureka.server.EnableEurekaServer;
        
        @EnableEurekaServer
        @SpringBootApplication
        public class DiscoveryServiceApplication {
        	
        	public static void main(String[] args) {
        		SpringApplication.run(DiscoveryServiceApplication.class, args);
        	}
        }    \end{lstlisting}  
    
    
            \chapter{Microservice ProxyService}
             \section*{ProxyServiceApplication }
    \begin{lstlisting}
    
    package tn.insat.tpmicro.discoveryservice;
    
    import org.springframework.boot.SpringApplication;
    import org.springframework.boot.autoconfigure.SpringBootApplication;
    import org.springframework.cloud.netflix.eureka.server.EnableEurekaServer;
    
    @EnableZuulProxy
    @EnableDiscoveryClient
    @SpringBootApplication
    public class ProxyServiceApplication {
    
    public static void main(String[] args) {
    SpringApplication.run(ProxyServiceApplication.class, args);
    }
    }    \end{lstlisting}  
    
    
    
    
    
    
    
    
    
    
  \chapter{L'organigramme de la CNR}
    
\begin{figure}[H]
	\centering
	\includegraphics[width=1.1\linewidth,height=0.568\paperheight]{images/orgpdf}
	\caption{L'organigramme de la CNR}
	\label{fig:orgpdf}
\end{figure}

\end{document}
