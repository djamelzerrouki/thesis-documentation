
  
 	 \section{Introduction}
 	  
    
    Le cloud computing, traduit le plus souvent en français par " informatique dans les nuages", " informatique dématérialisée " ou encore " infonuagique ", est un domaine qui regroupe un ensemble de techniques et de pratiques consistant à accéder, en libre-service, à du matériel ou à des logiciels informatiques, à travers une infrastructure réseau (Internet). Ce concept rend possible la distribution des ressources informatiques sous forme de services pour lesquels l'utilisateur paie uniquement pour ce qu'il utilise. Ces services peuvent être utilisés pour exécuter des applications scientifiques et commerciales, souvent modélisées sous forme de workflows.
     
    Ce chapitre présente une introduction au cloud computing et au workflow, nécessaire pour la compréhension générale de ce rapport.
    
    Tout d’abord, nous présentons dans la section 1.2 une introduction au paradigme du cloud computing. Nous donnons un aperçu général du cloud computing, y compris sa définition, ses caractéristiques principales et une comparaison avec les technologies connexes. Nous présentons les différents modèles de service, les différents modèles de déploiement, ainsi que les différents acteurs du cloud computing. Nous résumons quelques challenges de recherche en cloud computing. Par la suite, nous présentons, dans la section 1.3, une introduction au workflow et systèmes de gestion de workflow. Nous donnons le concept du workflow, sa définition, et l’architecture de référence d’un système de gestion de workflows. Nous énumérons quelques systèmes de gestion de workflows existant dans les grilles et clouds et, finalement, nous résumons l’intérêt  du cloud pour les workflows.
    
    \section{cloud computing}
    \subsection{Concept du cloud computing}