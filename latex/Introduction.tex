\chapter*{Introduction Générale }


\section*{Contexte et Motivation} 
Le Cloud computing  présente une technologie prometteuse qui facilite l'exécution des applications. Il fournit des services flexibles et évolutifs, à la demande des utilisateurs, via un modèle de paiement à l'usage. Généralement, il peut fournir quatre types de services: SaaS (Software as a Service), PaaS (Platform as a Service), IaaS (Infrastructure as a Service) et HaaS (Hardware as a Service). Ces services sont offerts avec différents niveaux de qualité de service afin de répondre aux besoins spécifiques de différents utilisateurs. Bien que de nombreux services de Cloud computing  ont des fonctionnalités similaires (par exemple des services de calculs, services de stockages, services  réseaux, etc.),

Parmi les  applications, qui peuvent être  utilisées en tant que service Cloud, nous trouvons celles de flux de travail (Workflow)  qui permettent l'exécution et le fonctionnement automatisés des processus d'organisation pour formaliser, normaliser et accélérer le traitement des documents et autres tâches.




\section*{Problématique}

Les entreprises modernes doivent faire preuve de capacités d'adaptation très importantes et d'une pratique du management et de l'organisation aussi souple que durable dans le temps.  La technologie du Workflow est née du besoin qu'ont les entreprises d'améliorer leurs performances, notamment grâce à une mâîtrise de l'information, une meilleure coopération des acteurs et une optimisation des processus opérationnels et métiers. Les solutions Workflow n’arrivent pas encore à démarrer en Algérie. C’est la raison pour laquelle nous  proposons un Workflow pour le traitement des dossiers administratifs notamment dans un environnement Cloud.


 
\section*{Objectifs}


\begin{enumerate}
\item Conception et implémentation d'un service Cloud.

\item Développement d'un service Workflow administratif  pour l'environnement Cloud notamment pour réduit le temps de traitement des dossiers.  

\item Développement d'un système générique qui peut être exécuté par plusieurs  organisations. 
\end{enumerate}



\section*{Organisation du mémoire}

Nous avons organisé notre mémoire comme suit : 
\subsection*{Chapitre 1: Concepts fondamentaux de Cloud et de Workflow} 

Ce chapitre vise à présenter les concepts fondamentaux de Cloud computing  et de Workflow 
\subsection*{Chapitre 2: Analyse et Conception} 
A travers ce chapitre, nous avons clarifié et détaillé les besoins à travers différents diagrammes. 
\subsection*{Chapitre 3: Réalisation}


Ce chapitre, décrit le choix des techniques et les outils de développement de notre solution, l’architecture technique, l’architecture du code et la configuration de application 
