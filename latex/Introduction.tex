\chapter*{Introduction Général }


\section*{Contexte et Motivation} 
Le cloud computing  présente une technologie prometteuse qui facilite l'exécution des applications . Il fournit des services flexibles et évolutifs, à la demande des utilisateurs, via un modèle de paiement à l'usage. Généralement, il peut fournir quatre types de services: SaaS (Software as a Service), PaaS (Platform as a Service), IaaS (Infrastructure as a Service) et HaaS (Hardware as a Service). Ces services sont offerts avec différents niveaux de qualité de service afin de répondre aux besoins spécifiques de différents utilisateurs. Bien que de nombreux services de cloud computing  ont des fonctionnalités similaires (par exemple des services de calculs, services de stockages, services  réseaux, etc.),

Parmi les  applications, qui peuvent être  utilisées en tant que service cloud, nous trouvons celles de flux de travail (Workflow)  qui permettent l'exécution et le fonctionnement automatisés des processus d'organisation pour formaliser, normaliser et accélérer le traitement des documents et autres tâches.




\section*{Problématique}

Les entreprises modernes doivent faire preuve de capacités d'adaptation très importantes et d'une pratique du management et de l'organisation aussi souple que durable dans le temps.  La technologie du Workflow est née du besoin qu'ont les entreprises d'améliorer leurs performances, notamment grâce à une mâîtrise de l'information, une meilleure coopération des acteurs et une optimisation des processus opérationnels et métiers. Les solutions Workflow n’arrive pas encore à démarrer en Algérie. C’est la raison pour laquelle nous  proposons un Workflow pour le traitement des dossiers administratifs notament dans un envirenement cloud.


 
\section*{Objectifs}


conception d'un service cloud .
Devlopment d'un service cloud pour le cloud qui peuvent execute par pluser orrganisation 

\section*{Organisation du mémoire}